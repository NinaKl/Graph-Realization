In this paper we initiate the study of the temporal graph realization problem with respect to the fastest path durations among its vertices, while we focus on periodic temporal graphs. Given an $n \times n$ matrix $D$ and a $\Delta \in \mathbb{N}$, the goal is to construct a $\Delta$-periodic temporal graph with $n$ vertices such that the duration of a fastest path from $v_i$ to $v_j$ is equal to $D_{i,j}$, or to decide that such a temporal graph does not exist. The variations of the problem on static graphs has been well studied and understood since the 1960's (e.g. [Erdos and Gallai, 1960], [Hakimi and Yau, 1965]).

As it turns out, the periodic temporal graph realization problem has a very different computational complexity behavior than its static (i.e. non-temporal) counterpart. First we show that the problem is NP-hard in general, but polynomial-time solvable if the so-called underlying graph is a tree. Building upon those results, we investigate its parameterized computational complexity with respect to structural parameters of the underlying static graph which measure the ``tree-likeness''. We prove a tight classification between such parameters that allow fixed-parameter tractability (FPT) and those which imply W[1]-hardness. We show that our problem is W[1]-hard when parameterized by the feedback vertex number (and therefore also any smaller parameter such as treewidth, degeneracy, and cliquewidth) of the underlying graph, while we show that it is in FPT when parameterized by the feedback edge number (and therefore also any larger parameter such as maximum leaf number) of the underlying graph.