In this paper we initiate the study of the \emph{temporal graph realization} problem with respect to the fastest path durations among its vertices, while we focus on periodic temporal graphs. Given an $n \times n$ matrix $D$ and a $\Delta \in \mathbb{N}$, the goal is to construct a $\Delta$-periodic temporal graph with $n$ vertices such that the duration of a \emph{fastest path} from $v_i$ to $v_j$ is equal to $D_{i,j}$, or to decide that such a temporal graph does not exist. The variations of the problem on static graphs has been well studied and understood since the 1960's, and this area of research remains active until nowadays. 

As it turns out, the periodic temporal graph realization problem has a very different computational complexity behavior than its static (i.e. non-temporal) counterpart. First we show that the problem is NP-hard in general, but polynomial-time solvable if the so-called underlying graph is a tree or a cycle. Building upon those results, we investigate its parameterized computational complexity with respect to structural parameters of the underlying static graph which measure the ``tree-likeness''. For those parameters, we essentially give a tight classification between parameters that allow for tractability (in the FPT sense) and parameters that presumably do not. We show that our problem is W[1]-hard when parameterized by the \emph{feedback vertex number} of the underlying graph, while we show that it is in FPT when parameterized by the \emph{feedback edge number} of the underlying graph. 