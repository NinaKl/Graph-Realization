\documentclass[11pt,a4paper]{article}

\usepackage[english]{babel}

\usepackage{fullpage}

\usepackage{amsmath}
\let\proof\relax
\let\endproof\relax


\usepackage{amsthm}
\usepackage{amssymb}
\usepackage{amsfonts}
\usepackage{mathrsfs}
\usepackage{wrapfig} %wrap the text around the picture

\usepackage{thmtools} 
\usepackage{thm-restate}

\usepackage{caption} %for subfigure - join multiple figures and add captions
\usepackage{subcaption}
\usepackage{changepage} %inside figure we can put \begin{adjustwidth}{-1cm}{-1cm} \end{adjustwidth} and we can ignore the margins
\usepackage{mathtools} % write text [under]{over} arrow $\xrightarrow[\text{world}]{\text{hello}}$

\usepackage{graphicx}
\graphicspath{{Figures/}}

%\usepackage{enumitem} %enumerate 
\usepackage{enumerate}
\usepackage{todonotes}
\newcommand{\todonkl}[2][]{\todo[color=red!100!green!33,#1]{NKL:\\ #2}} %todo note that starts with NKL:
\setlength{\marginparwidth}{3.7cm} %setting the length of todonote

\newtheorem{theorem}{Theorem}
\newtheorem{observation}{Observation}
\newtheorem{lemma}[theorem]{Lemma}
\newtheorem{corollary}[theorem]{Corollary}
\newtheorem{claim}[theorem]{Claim}
\newtheorem{conjecture}[theorem]{Conjecture}
\theoremstyle{remark}
\newtheorem{remark}[theorem]{Remark}
\theoremstyle{definition}
\newtheorem{definition}[theorem]{Definition}
\newtheorem{example}[definition]{Example}
\usepackage{comment}

\usepackage{algorithm}
\usepackage[noend]{algpseudocode}

%font encoding
\usepackage[T1]{fontenc}
\usepackage[utf8]{inputenc}
\usepackage{lmodern}

\usepackage{hyperref}

\usepackage[capitalise,nameinlink, noabbrev]{cleveref}
\crefname{claim}{Claim}{Claims}
\crefname{observation}{Observation}{Observations}

\usepackage{authblk}

%\usepackage{natbib}

\newcommand{\ie}{i.\,e.,\ }
\newcommand{\st}{s.\,t.,\ }
\newcommand{\NP}{\textrm{NP}}
\newcommand{\APX}{\textrm{APX}}
\newcommand{\FPT}{\textrm{FPT}}
\newcommand{\XP}{\textrm{XP}}

\newcommand{\NN}{\mathbb{N}}
\newcommand{\ZZ}{\mathbb{Z}}
\newcommand{\RR}{\mathbb{R}}
\newcommand{\QQ}{\mathbb{Q}}


%%%%%%%%%%%%%%%%%%%%%%%%%%%%%%%% Defining a problem
%%%%%%%%%%%%%%%%%%\problemdef{NAME} {Input} {Output}
\usepackage{tabularx}
\newcommand{\problemdef}[3]{
	\begin{center}
		\begin{minipage}{0.95\textwidth}
			\noindent
			#1
			\vspace{5pt}\\
			\setlength{\tabcolsep}{3pt}
			\begin{tabularx}{\textwidth}{@{}lX@{}}
				\textbf{Input:}& #2 \\
				\textbf{Question:}& #3
			\end{tabularx}
		\end{minipage}
	\end{center}
}


%%%%%%%%%%%%%%%%%%%%%%%%%%%%%%%%%% PROBLEMS NAMES
%% One label per edge, period Delta, exact realization
\newcommand{\deltaExactLong}{\textsc{Simple $\Delta$-periodic Exact Temporal Graph Realization}}
\newcommand{\deltaExact}{\textsc{Simple $\Delta$-TGR}}

%% k labels per edge, period Delta, exact realization
\newcommand{\kDeltaExactLong}{\textsc{$(k,\Delta)$-periodic Exact Temporal Graph Realization}}
\newcommand{\kDeltaExact}{\textsc{$(k,\Delta)$-TGR}}

%% One label per edge, period Delta, upper-bounded realization
\newcommand{\kDeltaUpperBoundLong}{\textsc{Simple $\Delta$-periodic Upper-Bounded Temporal Graph Realization}}
%\newcommand{\deltaUpperBound}{}

%% k labels per edge, period Delta, lower-bound realization
\newcommand{\kDeltaLowerBoundLong}{\textsc{Simple $\Delta$-periodic Lower-Bounded Temporal Graph Realization}}


\title{Realizing Temporal Graphs}
\author{Nina Klobas, George B. Mertzios, Hendrik Molter, Paul G. Spirakis}


\bibliographystyle{abbrv}
\begin{document}
	\maketitle

\section{Ideas}
\begin{itemize}
    \item Symmetric distance matrix $D$: polytime?
    \item Use $|D|_\infty$ (maximum value in $D$) + max degree $\Delta$ as parameter $\rightarrow$ should give FPT
    \item FPT with feedback edge number
    \item use as parameter $\max_{i,j} |D_{i,j}-D_{j,i}|$
    \item approximation (additive or multiplicative), maybe with graph as input
\end{itemize}

\section{Preliminaries}

Let $G=(V,E)$ and $k,\Delta\in \mathbb{N}$, and let $\lambda: E \rightarrow \{1,2,\ldots,\Delta\}^k$ be an edge-labeling function that assigns to every edge of $G$ exactly $k$ different labels from $\{1,\ldots,\Delta\}$. 
Then we denote by $(G,\lambda,\Delta)$ the \emph{$\Delta$-periodic temporal graph} $(G,\lambda_{\Delta})$, where for every edge $e\in E$ we have $\lambda_{\Delta}(e)=\{i\Delta + x : i\geq 0, x\in \lambda(e)\}$. 
In this case we call $\lambda$ a \emph{$\Delta$-periodic labeling} of $G$. If $k=1$, we call $\lambda$ a \emph{simple $\Delta$-periodic labeling} of $G$.

\begin{definition}[temporal path] \label{def:temporalPath+Duration}
Let $(G,\lambda)$ be a temporal graph. A \emph{temporal path} $P$ from $u=u_0$ to $v=u_k$ in $(G,\lambda)$ is a sequence $(u_0 u_1,t_1),(u_1 u_2,t_2),\ldots,(u_{k-1} u_k,t_k)$, where $(u_0,u_1,\ldots,u_k)$ is a path in the underlying static graph $G$, $t_i\in \lambda(u_{i-1}u_i)$ for every $i=1,\ldots,k$, and $t_1<t_2<\ldots<t_k$. 
The \emph{duration} of this temporal path $P$ from $u$ to $v$ is $d(P)=t_k - t_1 + 1$.
A \emph{fastest} temporal path $P$ from $u$ to $v$ in $(G,\lambda)$ is a temporal path from $u$ to $v$ with the smallest possible duration.
The duration of the \emph{fastest} temporal path from $u$ to $v$ is denoted by $d(u,v)$.
\end{definition}

	
	
\problemdef{\deltaExactLong\ (\deltaExact)}
{An $n \times n$ matrix $D$.}
{Does there exist a graph $G=(V,E)$ with vertices $\{v_1,\ldots,v_{n}\}$ 
and a simple $\Delta$-periodic labeling $\lambda: E \rightarrow \{1,2,\ldots,\Delta\}$ such that, 
for every $i,j$, the duration of the fastest temporal path from $v_i$ to $v_j$ in the $\Delta$-periodic temporal graph $(G,\lambda,\Delta)$ is equal to $D_{i,j}$.}

\problemdef{\kDeltaExactLong\ (\kDeltaExact)}
{An $n \times n$ matrix $D$ of positive integers.}
{Does there exist a graph $G=(V,E)$ with vertices $\{v_1,\ldots,v_{n}\}$ 
and a $\Delta$-periodic labeling $\lambda: E \rightarrow \{1,2,\ldots,\Delta\}^k$ such that, 
for every $i,j$, the duration of the fastest temporal path from $v_i$ to $v_j$ in the $\Delta$-periodic temporal graph $(G,\lambda,\Delta)$ is equal to $D_{i,j}$.}

\problemdef{\kDeltaUpperBoundLong}
{An $n \times n$ matrix $D$ of positive integers and an integer $k\in \mathbb{N}$.}
{Does there exist a graph $G=(V,E)$ with vertices $\{v_1,\ldots,v_{n}\}$ with $n+k-1$ edges 
and a $\Delta$-periodic labeling $\lambda: E \rightarrow \{1,2,\ldots,\Delta\}$ such that, 
for every $i,j$, the duration of the fastest temporal path from $v_i$ to $v_j$ in the $\Delta$-periodic temporal graph $(G,\lambda,\Delta)$ is \emph{at most} $D_{i,j}$.}

\problemdef{\kDeltaLowerBoundLong}
{An $n \times n$ matrix $D$ of positive integers and an integer $k\in \mathbb{N}$.}
{Does there exist a graph $G=(V,E)$ on the vertices $\{v_1,\ldots,v_{n}\}$ with $n+k-1$ edges 
and a $\Delta$-periodic labeling $\lambda: E \rightarrow \{1,2,\ldots,\Delta\}$ such that, 
for every $i,j$, the duration of the fastest temporal path from $v_i$ to $v_j$ in the $\Delta$-periodic temporal graph $(G,\lambda,\Delta)$ is \emph{at least} $D_{i,j}$.}
\todo[inline]{NOT TEMPORALLY INTERESTING: the difficulty is to find a static graph with equal labels}

\todo[inline]{The matrix $D$ is m x m, and capturing the distances from an edge e to an edge f (zoom 14 Nov 22, GH).}
\todo[inline]{Check if the reduction works for the UPPER BOUNDED version (seems to work)}
\todo[inline]{how to extend the poly algorithms of cycles? Easy candidate: FES, FVS, series-parallel graphs, or tw=2 graphs}
\todo[inline]{FVS, $c$-factor approximation, hardness of approximation for $c<1.5$}

Let $v_1,v_2,\ldots,v_n$ be the $n$ vertices of the graph $G$ that we are looking for in the above problems. 
For simplicity of the presentation (and with a slight abuse of notation), we may refer during the paper to the entry $D_{i,j}$ of the matrix $D$ as 
$D_{a,b}$, where $a=v_i$ and $b=v_j$. That is, we might put as indices of the matrix $D$ the corresponding vertices of $G$ instead of their numbering.


\begin{definition}[$D$-increasing temporal path]
Let $(G,\lambda,\Delta)$ be a $\Delta$-periodic temporal graph on $n$ vertices, and let $D$ be an $n\times n $ matrix of positive integers. 
Let $P$ be a temporal path from $u_0$ to $u_k$ on the vertices $u_0,u_1,\ldots,u_k$, and let $P^\textsc{ST}=(u_0,u_1,\ldots,u_k)$ be the underlying \emph{static} path in $G$.
Then $P$ is \emph{$D$-increasing} in $G$ if $0=D_{u_0,u_0}<D_{u_0,u_1}<D_{u_0,u_2}< \ldots < D_{u_0,u_k}$.
\end{definition}


\begin{observation}
Let $D$ be a \textsc{YES}-instance of \textsc{$(k,\Delta)$-TGR}, and let 
$(G,\lambda,\Delta)$ be a certificate for $D$, where $\lambda:E\rightarrow \{1,2,\ldots,\Delta\}^k$ is a $\Delta$-periodic labeling of $G$.
Then, for every pair $u,v$ of vertices of $G$, every fastest temporal path from $u$ to $v$ is $D$-increasing.
\end{observation}

Let $(G,\lambda)$ be a temporal graph with  $n = |V(G)|$ and  $m=|E(G)|$.
Let $v \in V(G)$ be an arbitrary vertex of $G$, denote with $deg_{max}$ the maximum degree of all vertices in $V(G)$ and let $S_v$ be the set of time edges in $(G,\lambda)$, incident to $v$, \ie $S_v = \{(uv, \lambda(uv)) | uv \in E(G)\}$.
In their work Wo et.~al. \cite{Wu2016Efficient} provide an algorithm that computes the duration of the fastest temporal path from a source vertex $v$ to all other vertices in $G$ in
$O(n + m \log c)$ time, where $c = \min \{|S_v|, deg_{max}\}$.
From this result, we can deduce the following. 
\begin{lemma} \label{lemma:calculatingD}
Let $(G, \lambda, \Delta)$ be a $\Delta$-periodic temporal graph, where $\lambda$ assigns exactly $k$ labels to every edge of $G$. 
Denote with $deg_{max}$ the maximum degree of vertices in $G$, let $deg(v_i)$ be the degree of vertex $v_i \in V(G)$,
and let $c_{v_i} = \min \{ k \cdot deg(v_i), deg_{max}\}$.\\
To calculate the durations of all fastest temporal paths from a fixed vertex $v \in V(G)$ to all other vertices, it takes $O(n + m \log c_v)$ time. \\
To calculate the durations of the fastest temporal paths among all pairs of vertices we then need $O(n^2 + m \sum_{v \in V(G)} \log c_v)$ time.
\end{lemma}

Observe that $O(n + m \log c_v) = O(n + m \log deg_{max}) = O(n + m \log n) = O(n^2 \log n)$ and similarly
$O(n^2 + m \sum_{v \in V(G)} \log c_v) = O(n^2 + mn \log deg_{max}) = O(n + mn \log n) = O(n^3 \log n)$.
Since an instance $I$ of \kDeltaExact\ problem is an $n \times n$ matrix $D$, the running time of calculating one row of $D$ (\ie the fastest temporal paths from a single vertex to all others) takes $O(|I|\log \sqrt{|I|})$-time and calculating the whole matrix $D$ (\ie the fastest temporal paths among all pairs of vertices) takes $O(|I|^{3/2} \log \sqrt{|I|})$ time.
Therefore, from now on we say that calculating (a row of) $D$ takes polynomial time.

%\section{Results}

\section{
\texorpdfstring{\deltaExact\ } {Simple Delta-TGR}}

In this section we study the hardness of \deltaExact. We first start with defining certain notions, that will be of use when solving the problem.

    \begin{definition}[Travel delays]
    Let $(G, \lambda, \Delta)$ be a temporal graph satisfying conditions of problem \deltaExact.
    Let $e_1=uv$ and $e_2=vz$ be two incident edges in $G$ with $e_1 \cap e_2 = v$.
    We define the \emph{travel delay} from $u$ to $z$ at vertex $v$, denoted with $\tau_v^{uz}$,
    as the difference of the labels of $e_2$ and $e_1$, where we subtract the value of the label of $e_1$ from the label of $e_2$,  modulo~$\Delta$.
    More specifically:
    \begin{equation}\label{eq:def-VertexWaitingTime}
       \tau_v^{uz} \equiv \lambda (e_2) - \lambda(e_1) \pmod \Delta.
    \end{equation}
    Similarly, $\tau_v^{zu} \equiv \lambda (e_1) - \lambda(e_2) \pmod \Delta$.
    \end{definition}
    Intuitively, the value of $\tau_v^{uz}$ represents how long a temporal path waits at vertex $v$ when first taking edge $e_1=uv$ and then edge $e_2 = vz$.

    From the above definition and the definition of the duration of the temporal path $P$ we get the following two observations.
    \begin{observation}\label{obs:durationPwithWaitingTimes}
        Let $P = (v_0, v_1, \dots, v_p)$ be the temporal path from $v_0$ to $v_p$..
        Then $d(P) = \sum_{i = 1}^{p-1} \tau_{v_i}^{v_{i-1}v_i} + 1 $.
    \end{observation}
    \begin{proof}
        For the simplicity of the proof denote $t_i = \lambda(v_{i-1}v_i)$, and suppose that $t_i \leq t_{i+1}$, for all $i \in \{1,2,3,\dots,p\}$.
        Then
        \begin{align*}
        \sum_{i = 1}^{p-1} \tau_{v_i}^{v_{i-1}v_i} + 1  
        &= \sum _{i = 1}^{p-1} (t_{i+1} - t_i) + 1 \\
        & = (t_2 - t_1) + (t_3  t_2) + \cdots + (t_p - t_{p-1}) + 1  \\
        & = t_{p-1} - t_1 + 1\\
        & = d(P)
        \end{align*}
        Now in the case when $t_i > t_{i+1}$ we get that $\tau_{v_i}^{v_{i-1}v_{i+1}} = \Delta + t_{i+1} - t_i$.
        At the end this still results in the correct duration as the last time we traverse the path $P$ is not exactly $t_p$ but $k \lambda + t_p$, for some $k$.
    \end{proof}
    We also get the following.
    \begin{observation}\label{obs:travel-delays-both-directions}
    Let $(G, \lambda, \Delta)$ be a temporal graph satisfying conditions of problem \deltaExact.
    For any two incident edges $e_1 = uv$ and $e_2 = vz$ on vertices $u,v,z \in V$, with $e_1 \cap e_2 = v$, we have $\tau_v^{zu} = \Delta - \tau_v^{uz} \pmod \Delta$.
    \end{observation}
    
    \begin{proof}
        Let $e_1 = uv$ and $e_2 = vz$ be two edges in $G$ for which $e_1 \cap e_2 = v$. 
        By the definition $\tau_v^{uz} \equiv \lambda (e_2) - \lambda(e_1) \pmod \Delta$ and $\tau_v^{zu} \equiv \lambda (e_1) - \lambda(e_2) \pmod \Delta$.
        Summing now both equations we get $\tau_v^{uz} + \tau_v^{zu} \equiv \lambda(e_2) - \lambda(e_1) + \lambda (e_1) - \lambda(e_2) \pmod \Delta$, and therefore $\tau_v^{uz} + \tau_v^{zu} \equiv 0 \pmod \Delta$, which is equivalent as saying $\tau_v^{uz} \equiv - \tau_v^{zu} \pmod \Delta$ or $\tau_v^{zu} = \Delta - \tau_v^{uz} \pmod \Delta$.
    \end{proof}

\begin{claim} \label{claim:unique-fastest-path-allDelays}
    Let $(G, \lambda, \Delta)$ be a temporal graph satisfying conditions of problem \deltaExact,
    and let $P=P_{1,k}$ be a fastest temporal path from $u=v_1$ to $v=v_k$ on $k$ vertices $v_1,v_2,\dots,v_k$.
    Let us denote with $P_{1,i}$ the sub-path of temporal path $P_{1,k}$, that starts at $v_1$ and finishes at $v_i$.
    Suppose that $P_{1,i}$ is also the fastest temporal path from $u=v_1$ to any other vertex $v_i$ in $P$.
    Then we can determine travel delays on $P$ using the following equation
    \begin{equation}\label{eq:trav-delays-paths}
        \tau_{v_i}^{v_{i-1},v_{i+1}} = D_{1,i+1} - D_{1,i},
    \end{equation}
    for all $i \in \{2,3, \dots, k-1\}$.
\end{claim}

\begin{proof}
    Let $P$ be a fastest temporal path from $v_1$ to $v_k$ with the properties from the claim, and let $v_i$ be an arbitrary vertex in $P \setminus \{v_1,v_k\}$.
    Using the properties of fastest paths and the definition of duration, we can rewrite \cref{eq:trav-delays-paths} as follows
    \begin{align*}
        \tau_{v_i}^{v_{i-1},v_{i+1}} & = D_{1,i+1} - D_{1,i} =  d(P_{1,i+1}) - d(P_{1,i}) \\
        & \equiv 
        \left(\lambda(v_{i}v_{i+1}) - \lambda(v_1v_2) + 1\right) -  
        \left(\lambda(v_{i-1}v_{i}) - \lambda(v_1v_2) + 1\right) \pmod \Delta \\
        & \equiv \lambda(v_{i}v_{i+1}) - \lambda(v_{i-1}v_{i}) \pmod \Delta,
        \end{align*}
    which is exactly the definition of $\tau_{v_i}^{v_{i-1},v_{i+1}}$.
\end{proof}

\subsection{
\texorpdfstring{\deltaExact\ } {Simple Delta-TGR}
is polynomial-time solvable on trees and paths}

Let $D$ be a matrix from \deltaExact, when the underlying graph $G$ of $D$ is a tree on $n$ vertices $\{v_1, v_2, \dots, v_n\}$.
Let $v_i,v_j$ be two arbitrary vertices in $G$. Then we know that there exists a unique (static) path $P$ among them.
Consequently, it follows that the temporal paths $P_{i,j}$ from $v_i$ to $v_j$ and $P_{j,i}$ from $v_j$ to $v_i$ are also unique, up to modulo of the period $\Delta$ of the labeling $\lambda$,
and therefore are the fastest.
Then $D$ is of the following form:
\begin{equation*}
    D_{i,j} =
    \begin{cases}
    0 & \text{if $i = j$}, \\
    1 & \text{if $v_i$ and $v_j$ are neighbours in G}\\
    d(P_{i,j}) & \text{else}
    \end{cases},
\end{equation*}
where $d(P_{i,j})$ is the duration of the (unique) temporal path $P_{i,j}$ from $v_i$ to $v_j$.

\begin{observation}\label{obs:travel-delays-in-trees}
    Let $v_i,v_j$  be arbitrary two vertices in a tree $G$. 
    Since there is a unique temporal path $P_{i,j}$ from $v_i$ to $v_j$, it is also the fastest one, therefore $d(P_{i,j}) = D_{i,j}$. 
    Note, all other vertices $v' \in P_{i,j} \setminus \{v_i,v_j\}$ are reached form $v_i$ using a part of the path $P_{i,j}$.
    Now using \cref{claim:unique-fastest-path-allDelays}, we can determine the waiting times for all \emph{inner} vertices of the path $P_{i,j}$.
\end{observation}

\begin{theorem} \label{thm:deltaExact-PolyTimeTrees}
    \deltaExact\ can be solved in polynomial time on trees.
\end{theorem}

\begin{proof}
    Let $D$ be an input matrix for problem \deltaExact of dimension $n \times n$.
    Let us fix the vertices of the corresponding graph $G$ of $D$ as $v_1, v_2, \dots, v_n$, where vertex $v_i$ corresponds to the row and column $i$ of matrix $D$.
    This can be done in polynomial time as we need to loop through the matrix $D$ once and connect vertices $v_i, v_j$ for which $D_{i,j} = 1$. At the same time we also check if $D_{i,i} = 0$, for all $i \in [n]$.
    When $G$ is constructed we run DFS algorithm on it and check if it has no cycles.
    If at any step we encounter a problem, our algorithm stops and returns a negative answer.
    
    From now on we can assume that we know that the underlying graph $G$ of $D$ is a tree and we know how it looks like.
    For the next part of the algorithm we use \cref{obs:travel-delays-in-trees}.
    
    We pick an arbitrary vertex $v_i \in V(G)$ and check which vertex $v_j \in V(G)$ is furthest away from it, \ie we find a maximum element in the $i$-th row of the matrix $D$.
    We now take the unique path $P_{i,j}$ in $G$, which has to also be the fastest temporal path from $v_i$ to $v_j$, and using \cref{obs:travel-delays-in-trees} calculate waiting times at all inner vertices.
    We save those values in a matrix $T$, of size $n \times n \times n$, and mark vertices of the path  $P_{i,j}$ as visited.
    Matrix $T$ stores at the position $(k,j,\ell)$ the value corresponding to the travel delay at vertex $v_k$ when traveling from $v_{j}$ to $v_{\ell}$, \ie it stores the value $\tau_{v_k}^{v_{j}, v_\ell}$, where $v_j,v_\ell \in N(v_k)$. All other values of $T$ are set to \textsc{Null}.
    Now we repeat procedure, from vertex $v_i$, for vertices that are not marked as visited yet, \ie vertices in $V \setminus P_{i,j}$.
    We find a vertex in $V \setminus P_{i,j}$ that is furthest away from $v_i$ and repeat the procedure.
    When we have exhausted the $i$-th row of $D$,
    \ie vertex $v_i$ now reaches every vertex of $G$,
    we continue and repeat the procedure for all other vertices.
    If at any point we get two different values for the same travel delay at a specific vertex, then we stop with the algorithm and return the negative answer.
    If the above procedure finishes successfully we get the matrix $T$ with travel delays for all vertices in $G$, of degree at least $2$.
    The above calculation is performed in polynomial time, as for every vertex $v_i$ we inspect the whole graph once.
    \begin{claim}\label{claim:matrixT-travelDelays-tree}
        Matrix $T$ consists of travel delays of all vertices of degree at least $2$ in $G$.
    \end{claim}
    \begin{proof}[Proof of \cref{claim:matrixT-travelDelays-tree}]
        Note, by the definition of travel delays, a vertex of degree $1$ cannot have a travel delay.
        Suppose now that there is a vertex $v_i \in V(G)$ of degree at least $2$, for which our algorithm did not calculate its travel delay.
        Let $v_a, v_b$ be two arbitrary neighbors of $v_i$, \ie $v_a v_i, v_i v_b \in E(G)$.
        Since $G$ is a tree, the unique (and fastest) temporal path from $v_a$ to $v_b$ passes through $v_i$.
        When our algorithm was inspecting the row of $D$ corresponding to vertex $v_a$, it had to consider the temporal path from $v_a$ to $v_b$. 
        At this point, it calculated $\tau_{v_i}^{v_a,v_b}$. 
        Since this is true for any two $v_a, v_b \in N(v_i)$, it cannot happen that some travel delay at $v_i$ is not calculated.
        Since $v_i$ was an arbitrary vertex in $G$ of degree at least $2$, the claim holds.
    \end{proof}
    
    Now, given the matrix of travel delays $T$, we can find a labeling $\lambda$ that satisfies $D$.
    We start by fixing a label of one arbitrary edge as $a$, where $a \in [\Delta]$.
    Knowing the label of one edge, and all waiting times in $T$, we can uniquely determine the labels of all other edges.
    More specifically, if we know that $\lambda(v_i v_j) = a$, then for all $v_k \in N(v_i)$ (resp.~$v_{k'} \in N(v_j)$)
    the value $\lambda(v_iv_k) \equiv a + \tau_{v_i}^{v_j,v_k} \pmod \Delta $ (resp.~$\lambda(v_j v_{k'}) \equiv a + \tau_{v_j}^{v_i,v_{k'}} \pmod \Delta $).
    Since there are $\Delta$ options to fix the first label, we can find $\Delta$ different labelings satisfying $D$.
    Note, w.l.o.g. we can start determining the labeling $\lambda$ by setting $\lambda(v_1v_2) = 1$.
    It is not hard to see, that also the calculation of the labeling $\lambda$ takes polynomial time, as we have to traverse the graph exactly once, to successfully fix the labeling. Therefore, all together the whole algorithm is performed in polynomial time.
\end{proof}

\subsection{
\texorpdfstring{\deltaExact\ } {Simple Delta-TGR}
is polynomial-time solvable on cycles}

Let us observe some properties of the matrix $D$ from \deltaExact, when the underlying graph $G$ of $D$ is a cycle $C_n = \{v_1, v_2, \dots, v_n\}$  on $n$ vertices.
By the definition, each vertex is on distance $0$ from itself. This corresponds with all diagonal entries of $D$ being $0$.
Now, let us observe that each vertex $v_i$ has exactly two neighbours in $C_n$, namely $v_{i-1}$ and $v_{i+1}$,
therefore for all $i\in [n]$ we set $D_{i,i-1} = D_{i, i+1} = 1$, where indices are taken modulo $n$.
This results in the upper and lower diagonal of $D$ having all $1$'s, together with $D_{1,n} = D_{n,1} = 1$.
The matrix $D$ is of the following form
\begin{equation} \label{eq:D-matrixForCycles}
  D = 
  \begin{bmatrix}
    0 & 1 &  &  &  & 1\\
    1 & 0 & 1 &  &  & \\
    & 1 & 0 & 1 & & \\
    & & \ddots & \ddots & \ddots &   \\
    &   & &1 &  0 & 1\\ 
    1 &  &  &  &1 &  0 
  \end{bmatrix},
\end{equation}
where the empty entries consists of positive integers different than $1$.

Given a matrix in the input of \deltaExact\ we can check in $O(n^2)$ time if it is of correct form, by traversing it once. If it is not of correct form, our algorithm stops and returns the negative answer.
From now on, we assume, that the input matrix has the same form as $D$ from \cref{eq:D-matrixForCycles}.

Let $v_i$ be an arbitrary vertex in the cycle $C_n =\{v_1, v_2, \dots, v_n\}$.
Vertex $v_i$ can reach an arbitrary vertex $v_k \in C_n$ using a positive side of the cycle (\ie going from $v_i$ to $v_{i+1},  v_{i+2}$, etc.), 
and using the negative side of the cycle (\ie going from $v_i$ to $v_{i-1}, v_{i-2}$, etc.).
Let us denote with the $d^+(i,k)$ the duration of the temporal path from $v_i$ to $v_k$ using the positive side of the cycle,
and with $d^-(i,k)$, the duration of the temporal path from $v_i$ to $v_k$ using the negative side of the cycle.
Since these two are the only possible paths from $v_i$ to $v_k$ in $C_n$ we know that $D_{i,k} = min \{d^+(i,k), d^-(i,k)\}$.

\begin{claim} \label{claim:poly-delta-cycle-monotonicity}
    If vertex $v_i$ reaches vertex $v_j$ the fastest, using the positive (resp.~negative) side of the cycle, \ie $v_i, v_{i+1}, \dots, v_{j-1}, v_j$ (resp. $v_i, v_{i-1}, \dots, v_{j+1}, v_j$),
    then $v_i$ reaches all other vertices $v_k$, where $k \in \{i+1, i+2, \dots, j-1\}$ (resp.~ $k \in \{i-1, i-2, \dots, j+1\}$)
    using the same, positive (resp.~negative) side of the cycle,
    where the indices are taken modulo $n$.
\end{claim}
\begin{proof}
    Let $v_i, v_j$ be two arbitrary vertices in $C_n$ and suppose that $D_{i,j} = d^+(i,j)$,
    but there exists a vertex $v_k$ from $C_n$, where $k \in \{i+1, i+2, \dots, j-1\}$,
    for which $D_{i,k} \neq d^+(i,k)$.
    Then $d^-(i,k)$ passes through vertex $v_j$, and we can split the path from $v_i$ to $v_k$ into two pieces, 
    first from $v_i$ to $v_j$ and the second one from $v_k$ to $v_j$.
    So 
    \begin{equation} \label{eq:poly-delta-cycle-monotonicity}
        d^-(i,k) = d^-(i,j) + d^-(j,k)^*,
    \end{equation} where $d^-(j,k)^*$ is the duration of the path from $v_j$ to $v_k$, using the negative side of the cycle, with consideration that we come to vertex $v_j$ at time $d^-(i,j)$ and we potentially have to wait there for some positive amount of time, before we continue to $v_j$
    (\ie $d^-(j,k)^* \geq d^-(i,j) \geq D_{j,k}$ includes some waiting time at vertex $v_j$).
    By the assumption we know that $d^+(i,j) \leq d^-(i,j)$, so we can rewrite \cref{eq:poly-delta-cycle-monotonicity} as
    $d^-(i,k) \leq d^+(i,j) + d^-(j,k)^*$. This means that we can reach $v_k$ from $v_j$ by going from $v_i$ to $v_j$ using the positive side of the cycle, wait some time at $v_j$, before we continue back to $v_j$.
    In the above construction vertex $v_k$ is visited twice. By the definition, the fastest temporal path from $v_i$ to $v_k$ visits $v_k$ exactly once. Therefore we can stop at $v_k$ already when traveling from $v_i$ to $v_j$ the first time, using the positive side of the cycle.
    It follows that $D_{i,j} = d^+(i,j) \leq d^-(i,j)$, which contradicts with our assumption.
\end{proof}

Let $v_i$ and $v_j$ be two arbitrary vertices in the cycle $C_n$, for which $v_i v_j \notin E(C_n)$.
Suppose that $P^+_{i,j}$ (resp.~$P^-_{i,j})$ is the fastest temporal path from $v_i$ to $v_j$, using the positive (resp.~negative) side of the cycle, 
\ie $v_i, v_{i+1}, \dots, v_{j-1}, v_j$ (resp. $v_i, v_{i-1}, \dots, v_{j+1}, v_j$).
Then by \cref{claim:poly-delta-cycle-monotonicity} and \cref{claim:unique-fastest-path-allDelays},
we get that we can determine all travel delays at vertices of $P^+_{i,j} \setminus \{v_i, v_j\}$ (resp.~$P^-_{i,j} \setminus \{v_i, v_j\}$).


Let $v_i$ be an arbitrary vertex of $C_n$. Let us look at the row $i$ of the matrix $D$, which corresponds to the durations of fastest temporal paths from $v_i$ to all other vertices $v_j \in C_n$.
Using \cref{claim:poly-delta-cycle-monotonicity} we know that $v_i$ will reach some consecutive vertices $v_{i+1}, v_{i+2}, \dots, v_j$ the fastest, using the positive side of the cycle 
and $v_{i-1}, v_{i-2}, \dots, v_{j+1}$ the fastest, using the negative side of the cycle.
Suppose $v_j \in C_n$ is the last vertex $v_i$ reaches using the positive side of the cycle, and $v_{j+1}$ the last vertex that is reached using the negative side of the cycle.
Then we know that $D_{i,i+1} < D_{i,i+2} < \cdots< D_{i,j-1} < D_{i,j}$ and 
$D_{i,i-1} < D_{i,i-2} < \cdots < D_{i,j+2} < D_{i,j+1}$.
Note that it can happen that $v_j = v_{j+1}$, \ie to reach vertex $v_j$ from $v_i$ the fastest, we can use either positive or negative side of the cycle.
Using the above observations, every row $i$ ($i \in [n]$) of matrix $D$ has two (or one) maximum elements, one at position $j$ and the other at position $j+1$, where $j, j+1 \in [n]$ and the indices are considered modulo $n$.
Let us denote these two values as $m_i^1$ and $m_i^2$.
The row $i$ of $D$ is of the following form, it has a $0$ at the entry $i$, it has $1$ at entries $i-1, i+1$,
the values increase on the positions $i+1, i+2, i+3, \dots, j-1, j$ for some $j$ with value $m_i^1$, 
and on the other side, values increase on the positions $i-1, i-2, i-3, \dots, j+2, j+1$ for some $j+1$ with value $m_i^2$,
where indices are taken modulo $n$.

Knowing this, we can split the vertices $v_j \in V(C_n) \setminus \{v_i\}$ into two parts, ones that are reached from $v_i$ the fastest using the positive side of the cycle and ones that are reached using the negative side of the cycle.
To determine these two sets we do the following.
We fix a vertex $v_i \in C_n$ and check its corresponding row in the matrix $D$. 
We determine two max values $m_i^1$ and $m_i^2$ at positions $j$ and $j+1$ (modulo $n$), respectively,
for which it has to hold that
$D_{i,i+1} < D_{i,i+2} < \cdots< D_{i,j-1} < D_{i,j}=m_i^1$ and 
$D_{i,i-1} < D_{i,i-2} < \cdots < D_{i,j+2} < D_{i,j+1}=m_i^2$.
Note, it can also happen that $m_i^1 = m_i^2$. 
Now denote the path that uses the positive side of the cycle, from $v_i$ to $v_j$, as $P^+_i$ and
the path that uses the negative side of the cycle, from $v_i$ to $v_{j+1}$, as $P^-_i$.
By \cref{claim:poly-delta-cycle-monotonicity} and \cref{claim:unique-fastest-path-allDelays} we can calculate travel delays at every vertex $v_k \in C_n \setminus \{v_i, v_j, v_{j+1}\}$,
which we store in a list $T$ of length $n$,
where the entry at the position $k$ corresponds to the travel delay at vertex $v_k$ when traveling from $v_{k-1}$ to $v_{k+1}$.
Note, from \cref{obs:travel-delays-both-directions} it follows, that it is enough to store the value of the travel delay in one direction (\ie knowing $\tau_v^{u,w}$ we know also $\tau_v^{w,u}$).
We repeat the above procedure for all rows in the matrix $D$, \ie for all vertices $v_i \in C_n$.
Calculation in one row is performed in $O(n)$ time, repeating it for all rows we need $O(n^2)$ time.

To determine the labeling $\lambda$ satisfying the matrix $D$, we have to make sure that we have calculated travel delays for all vertices.

\begin{claim}
List $T$ of travel delays has non-empty values at all positions, \ie we have successfully calculated travel delays for all vertices.
\end{claim}
\begin{proof}
    Throughout the proof we
    denote with $d^+(i,j)$ the duration of the temporal path that starts at $v_i$ and finishes at $v_j$, that uses the positive side of the cycle $C_n$,
    and similarly with $d^-(i,j)$ the duration of the temporal path that starts at $v_i$ and finishes at $v_j$, that uses the negative side of the cycle $C_n$,
    where indices are taken modulo $n$.
    
    Suppose for the contradiction that the statement of the claim does not hold. Then there exists a vertex $v_i \in C_n$, for which we did not calculate its travel delay.
    Let $v_j \in C_n \setminus \{v_i\}$ be an arbitrary vertex. 
    Note that the only time we cannot calculate the travel delay at vertex $v_i$, when considering vertex $v_j$, is in this case when $v_i$ is one of the maximum elements in $D_j$, \ie $v_i$ is a vertex that is on a maximum duration from $v_j$.
    This has to hold for all vertices $v_j$, therefore $v_i$ has to be on the maximum distance from every vertex $v_j \in C_n \setminus \{v_i\}$.
    It then also has to hold for vertices $v_{i-1}, v_{i+1}$, that are neighbours of $v_i$.
    We know that in this case $d^+(i-1,i) = 1, d^-(i+1,i) = 1$.
    Since $v_i$ has to be on the maximum duration from both of them 
    (\ie is one of the two maximum values $m^1_{i-1}, m^2_{i-1}$ for vertex $v_{i-1}$ and one of the two maximum values $m^1_{i+1}, m^2_{i+1}$ for vertex $v_{i+1}$),
    we know that 
    \begin{align}\label{eq:cycle-travel-delays-all}
        & d^+(i+1,i-1) < d^-(i+1,i-1) & \text{and}\\
        & d^-(i-1,i+1) < d^+(i-1,i+1).
    \end{align}
    If this would not hold, the fastest path would go through $v_i$ and we would be able to calculate the travel delay at $v_i$.
    
    Denote the labels $\lambda(v_{i-1}v_i) = a, \lambda(v_{i}v_{i+1})=a'$, and $\lambda(v_{i-2}v_{i-1} = b, \lambda (v_{i+1}v_{i+2})=b'$, where $a,a',b,b' \in [\Delta]$.
    W.l.o.g. we can suppose that $a \geq a'$.
    Therefore, using the definition for the duration of temporal paths, we get that
    $d^+(i+1,i-1) = (k \delta + b) - b' + 1$, where $k$ is some non-negative integer, and $d^-(i+1,i-1) = a - a' + 1$.
    Using the first inequality from \cref{eq:cycle-travel-delays-all} we get that 
    $(k \delta + b) - b' +1 < a - a' + 1$ which is equivalent to $(k \delta + b) - b' < a - a'$, which can be true only when $k = 0$, but because the duration is positive we get that $b > b'$ and
    $b - b' < a - a'$.
    Now again using the definition of the duration, we get
    $d^+(i-1,i+1)= (k\delta + a') - a + 1$, where $k$ is some non-negative integer, but since $v_{i-1}v_i$ and $v_iv_{i+1}$ are incident edges, we know also that $k=1$, therefore $d^+(i-1,i+1)= \delta + a - a + 1$. 
    Again by the definition, $d^-(i-1,i+1) = (k' \delta + b') -b + 1$, for some non-negative integer $k'$.
    Using the second inequality from \cref{eq:cycle-travel-delays-all} we get that 
    $(k' \delta + b') -b + 1 < \delta + a - a + 1$.
    Which is equivalent to  $(k' \delta + b') -b < \delta + a' - a $. This can hold only when $k' = 1$, in that case we get $ \delta + b' -b < \delta + a' - a $ which is equivalent to $b' -b < a' - a$.
    This is in the contradiction with the inequality  $b - b' < a - a'$ as this is equivalent to $b'-b > a'-a$.
    
    Therefore it cannot happen that there is a vertex $v_i$ for which we cannot calculate its travel delay in $C_n$.
\end{proof}

All of the above observations imply the following result.
\begin{theorem}
    \deltaExact\ can be solved in polynomial time on cycles.
\end{theorem}

\begin{proof}
As stated above, we can determine travel delays at every vertex in $O(n^2)$ time. 
Once all of the delays are calculated, we have to only construct the labeling $\lambda$ that satisfies the matrix $D$. 
We start by fixing a label of one edge as $a$, where $a \in [\Delta]$.
Knowing the label of edge $v_1v_2$ and all waiting times in $T$, we can uniquely determine the labels of all other edges.
More specifically, if we know that $\lambda(v_i v_j) = a$, then for all $v_k \in N(v_i)$ (resp.~$v_{k'} \in N(v_j)$)
the value $\lambda(v_iv_k) \equiv a + \tau_{v_i}^{v_j,v_k} \pmod \Delta $ (resp.~$\lambda(v_j v_{k'}) \equiv a + \tau_{v_j}^{v_i,v_{k'}} \pmod \Delta $).
Since there are $\Delta$ options to fix the first label, we can find $\Delta$ different labelings satisfying $D$.
Note, w.l.o.g. we can start determining the labeling $\lambda$ by setting $\lambda(v_1v_2) = 1$.
\end{proof}

    
\section{Hardness of Exact Realization}

\todo[inline]{HM: For the following hardness I assume that we only consider strict temporal paths. (NK: Yes you are right, all paths are strict, sorry forgot to specify). \\
Furthermore, I can currently only make it work if we do not have periods ($\Delta=\infty$). I am not sure the idea also works if we have periods.}

\begin{theorem}
	\textsc{Exact Realization $(\infty)$} is \NP-hard.
\end{theorem}

\begin{proof}
	We present a polynomial-time reduction from the NP-hard problem 3-SAT~\cite{Karp1972Reducibility}. Here, we are given a formula $\phi$ in conjunctive normal form, where each clause contains exactly 3 literals (with three distinct variables). We construct an instance of \textsc{Exact Realization $(\infty)$} as follows.

 We start by describing the vertex set of the underlying graph $G$.
\begin{itemize}
\item For each variable $x$ in $\phi$, we create three variable vertices $x, x^T, x^F$.
\item For each clause $c$ in $\phi$, we create one clause vertex $c$.
\item We add one additional super vertex $v$.
\end{itemize}
Next, we describe the edge set of $G$.
\begin{itemize}
\item For each variable $x$ in $\phi$ we add the following five edges: 

$\{x, x^T\}, \{x, x^F\}, \{x^T, x^F\}, \{x^T, v\}, \{x^F,v\}$.
\item For each pair of variables $x,y$ in $\phi$ with $x\neq y$ we add the following four edges 

$\{x^T,y^T\},\{x^T,y^F\}, \{x^F,y^T\},\{x^F,y^F\}$.
\item For each clause $c$ in $\phi$ we add one edge for each literal. Let $x$ appear in $c$. If $X$ appears non-negated in $c$ we add edge $\{c,x^T\}$. If $x$ appears negated in $c$ we add edge $\{c, x^F\}$.
\end{itemize}
This finishes the construction of $G$.

Now we specify the distances between all vertex pairs. Naturally, the distance between all pairs of adjacent vertices is one.
\begin{itemize}
    %\item For each variable $x$ in $\phi$ we specify the following distances between the non-adjacent variable vertices:
    
    %$d(x_1,x_2)=2$. 
    %\item For pair of variable $x,y$ in $\phi$ with $x\neq y$ we specify the following distances:

    %$d(x_1,y^T)=d(x_1,y^F)=3$, $d(x_2,y^T)=d(x_2,y^F)=2$.
    \item For each variable $x$ in $\phi$ we specify the following distances to the super vertex $v$:

    $d(x,v)=2$. %, $d(x_2,v)=3$.

    \item For each clause $c$ in $\phi$ we specify the following distances to the super vertex $v$:

    $d(c,v)=2$
    \item Let $x$ be a variable that appears in clause $c$, then  we specify the following distances:

    $d(c,x)=2$.
    
    If $x$ appears non-negated in $c$ we specify the following distances:

    $d(c,x_F)=2$.

    If $x$ appears negated in $c$ we specify the following distances:

    $d(c,x_T)=2$.
    \item Let $x$ be a variable that does \emph{not} appear in clause $c$, then we specify the following distances:

    $d(c,x^T)=d(c,x^F)=2$.
\end{itemize}
All distances between non-adjacent vertex pairs that are not defined above are set to $\infty$.

This finishes the construction of the \textsc{Exact Realization $(\infty)$} which can clearly be done in polynomial time. In the remainder we show that it is a yes-instance if and only if $\phi$ is satisfiable.

$(\Rightarrow)$: Assume the constructed \textsc{Exact Realization $(\infty)$} is a yes-instance. Then there exist a label $\lambda(e)$ for each edge $e\in E(G)$ such that for each vertex pair $u,w$ in the temporal graph $(G,\lambda)$ we have that a fastest temporal path between from $u$ to $w$ has exactly duration $d(u,w)$. In particular, for any two vertices $u,w$ with $d(u,w)=\infty$ we have that there does not exist a temporal path from $u$ to $w$ in $(G,\lambda)$.

We construct a satisfying assignment for $\phi$ as follows. For each variable $x$, if $\lambda(\{x,x^T\})=\lambda(\{x^T,v\})$, then we set $x$ to \texttt{true}, otherwise we set $x$ to \texttt{false}.

To show that this yields a satisfying assignment, we need to show some following properties of $\lambda$.
First, observe that adding some integer $t$ to all time labels does not change the duration of any temporal path. Second, observe that if for two vertices $u,w$ we have that $d(u,w)$ equals the distance between $u$ and $w$ in $G$, then there is a shortest path from $u$ to $w$ in $G$ such that $\lambda$ puts consecutive time labels on the edges of that shortest path. 


Let $\lambda(\{x,x^T\})=t$ and $\lambda(\{x,x^F\})=t'$, for an arbitrary variable $x$. 
If both $\lambda(\{x^T,v\})\neq t+1$ and $\lambda(\{x^F,v\})\neq t'+1$, then $d(x,v)>2$, which is a contradiction. 
Thus, for every variable $x$, we have that $\lambda(\{x^T,v\})= t+1$ or $\lambda(\{x^F,v\})= t'+1$ (or both). 
In particular, this means that if $\lambda(\{x,x^F\})=\lambda(\{x^F,v\})$, then we set $x$ to \texttt{false}, since in this case $\lambda(\{x,x^T\})\neq\lambda(\{x^T,v\})$.

%Furthermore, we have that if $x$ appears non-negated in clause $c$, then $\lambda(\{c,x^T\})=t-1$, otherwise we have $d(c,x^T)>2$. Symmetrically, we have that if $x$ appears negated in clause $c$, then $\lambda(\{c,x^F\})=t'-1$, otherwise we have $d(c,x^F)>2$.

Now assume for contradiction that the described assignment is not satisfying. Then there exists a clause $c$ that is not satisfied. Recall that we require $d(c,v)=2$. Hence, we must have a temporal path consisting of two edges from $c$ to $v$ such that the two edges have consecutive labels. By construction of $G$ there are three candidates for such a path, one for each literal of $c$. 
Assume w.l.o.g\ that $x$ appears in $c$ non-negated (the case of a negated appearance of $x$ is symmetrical) and that the temporal path realizing $d(c,v)=2$ goes through vertex $x^T$. 
Then we have that $\lambda(\{c,x^T\})=\lambda(\{x^T,v\})-1$. Furthermore, since $d(c,x)=2$ we also have that $\lambda(\{c,x^T\})=\lambda(\{x,x^T\})-1$. It follows that $\lambda(\{x,x^T\})=\lambda(\{x^T,v\})$. However, this implies that $x$ is set to \texttt{true} in the satisfying assignment and thus the clause $c$ is satisfied, a contradiction. 


%Assume w.l.o.g.\ that variable $x$ appears non-negated in clause $c$. Then we have that $\{c,x^T\}\in E(G)$. Assume that $\lambda(\{c,x^T\})=t$. Recall that we have specified $d(c,x_2)=2$. Hence we have that $\lambda(\{x_2,x^T\})=t+1$.

$(\Leftarrow)$: Assume that $\phi$ is satisfiable. Then there exists a satisfying assignment for the variables in $\phi$.

We construct a labeling $\lambda$ as follows.
\begin{itemize}
    \item All edges incident with a clause vertex $c$ obtain label one.
    %\item For each variable $x$, we set $\lambda(\{x^T,x^F\})=1$.
    \item If variable $x$ is set to \texttt{true}, we set $\lambda(\{x^F,v\})=3$.
    \item If variable $x$ is set to \texttt{false}, we set $\lambda(\{x^T,v\})=3$.
    \item We set the labels of all other edges to two.
\end{itemize}

Next, we verify that all distances are realized.
\begin{itemize}
    \item For each variable $x$ in $\phi$ we have $d(x,v)=2$: 
    
    If $x$ is set to \texttt{true}, then there is a temporal path from $x$ to $v$ via $x^F$ such that $\lambda(\{x,x^F\})=2$ and $\lambda(\{x^F,v\})=3$. If $x$ is set to \texttt{false}, then there is a temporal path from $x$ to $v$ via $x^T$ such that $\lambda(\{x,x^T\})=2$ and $\lambda(\{x^T,v\})=3$.

    \item For each clause $c$ in $\phi$ we have that $d(c,v)=2$:

    Since we have a satisfying assignment there is a variable $x$ appearing in $c$ that is set to a truth-value that satisfies $c$. If $x$ appears non-negated in $c$ (and hence is set to \texttt{true}), then there is a temporal path from $c$ to $v$ through $x^T$ such that $\lambda(\{c,x^T\})=1$ and $\lambda(\{x^T,v\})=2$. If $x$ appears negated in $c$ (and hence is set to \texttt{false}), then there is a temporal path from $c$ to $v$ through $x^F$ such that $\lambda(\{c,x^F\})=1$ and $\lambda(\{x^F,v\})=2$.
    \item Let $x$ be a variable that appears in clause $c$.
    If $x$ appears non-negated in $c$ we have $d(c,x)=d(c,x_F)=2$:

    There is a temporal path from $c$ to $x$ via $x^T$ and also a temporal path from $c$ to $x^F$ via $x^T$ such that $\lambda(\{c,x^T\})=1$ and $\lambda(\{x,x^T\})=\lambda(\{x^T,x^F\})=2$.

    If $x$ appears negated in $c$ we have $d(c,x)=d(c,x_T)=2$:

    There is a temporal path from $c$ to $x$ via $x^F$ and also a temporal path from $c$ to $x^T$ via $x^F$ such that $\lambda(\{c,x^F\})=1$ and $\lambda(\{x,x^F\})=\lambda(\{x^T,x^F\})=2$.
    \item Let $x$ be a variable that does \emph{not} appear in clause $c$, then we have $d(c,x^T)=d(c,x^F)=2$:

    Let $y$ be a variable that appears non-negated in $c$ (the case where $y$ appears negated is symmetrical). Then there is a temporal path from $c$ to $x^T$ via $y^T$ and also a temporal path from $c$ to $x^F$ via $y^T$ such that $\lambda(\{c,y^T\})=1$ and $\lambda(\{y^T,x^T\})=\lambda(\{y^T,x^F\})=2$.
\end{itemize}
Lastly, we show that all non-adjacent vertex pairs $u,w$ with $d(u,w)=\infty$ are not temporally connected in $(G,\lambda)$.
\begin{itemize}
\item For all pairs of clause vertices $c,c'$ we have $d(c,c')=\infty$: 
\todo[inline]{New matrix entries D for period Delta: d(c,c')=Delta+1 (using three edges, with labels L, L+1, L).}

Since all edges incident with clause vertices have label one and clause vertices are pairwise non-adjacent, there cannot be a temporal path from one clause vertex to another.
\item For all variable vertices $x$ and clause vertices $c$ we have $d(x,c)=\infty$:
\todo[inline]{New matrix entries D for period Delta: if x appears (negated or not) in c, then d(x,c)=Delta (using two edges, with label L+1, L). Otherwise D(x,c)=2Delta (using three edges, with labels L, L, L-1).}

Since all edges incident with variable vertices $x$ have label two and all edges incident with clause vertices $c$ have label one, there cannot be a temporal path from a variable vertex $x$ to a clause vertex $c$.
\item For all pairs of variable vertices $x,y$ we have $d(x,y)=\infty$:
\todo[inline]{New matrix entries D for period Delta: d(x,y)=2Delta+1 (using three edges, each with the same label).}

Since all edges incident with variable vertices $x$ have label two and variable vertices $x$ are pairwise non-adjacent, there cannot be a temporal path from one variable vertex to another.
\item For all pairs of variable vertices $x,y$ we have $d(y^T,x)=d(y^F,x)=\infty$:
\todo[inline]{New matrix entries D for period Delta: $d(y^T,x)=d(y^F,x)=Delta+1$ (using two edges, with equal labels).}

Since all edges incident with variable vertices $x$ have label two and all edges incident with variable vertices $y^T, y^F$ have labels two or three and variable vertices $x$ are non-adjacent to variable vertices $y^T,y^F$, there cannot be a temporal path from a variable vertex $y^T$ or $y^F$ to variable vertex $x$.
\item For all pairs of variable vertices $x,y$ we have $d(x,y^T)=d(x,y^F)=\infty$:
\todo[inline]{New matrix entries D for period Delta: $d(x,y^T)=d(x,y^F)=Delta+1$ (using two edges, with equal labels).}

Observe that edges incident with $y^T, y^F$ that have label three connect $y^T, y^F$ to $v$. Furthermore, $v$ has distance two from $x$ in the undetermined graph $G$. Since all edges incident with variable vertices $x$ have label two, there is no temporal path starting at $x$ and arriving at $v$ at time two, that could then continue to $y_T$ or $y_F$ using an edge with label three. Hence, there is no temporal path from $x$ to $y^T, y^F$ using an edge with label three. Since all edges incident with $x$ have label two and variable vertices $x$ are non-adjacent to variable vertices $y^T,y^F$, there cannot be a temporal path from variable vertex $x$ to variable vertex $y^T$ or $y^F$.
\item For all variable vertices $x$ and clause vertices $c$ we have $d(v,x)=d(v,c)=\infty$:
\todo[inline]{New matrix entries D for period Delta: d(v,x)=Delta (using two edges, with labels L+1,L). 
New matrix entries D for period Delta: d(v,c)=Delta (using two edges, with labels L,L-1).}

All edges incident with $v$ have labels two or three. All edges incident with $x$ have label two and all edges incident with $c$ have label one. We immediately have that there is no temporal path from $v$ to $c$. Furthermore, since $x$ and $v$ are non-adjacent, we also have that there is no temporal path from $v$ to $x$.
\end{itemize}
\end{proof}


\section{W[1]-hardness}
\begin{theorem}\label{thm:W1wrtFVS}
    \deltaExactLong\ is W[1]-hard parameterized by the feedback vertex number.
\end{theorem}
\begin{proof}
    We present a parameterized reduction from the W[1]-hard problem \textsc{Multicolored Clique} parameterized by the number of colors~\cite{fellows2009multipleinterval}.  Here, given a $k$-partite graph $H=(W_1\uplus W_2 \uplus\ldots\uplus W_k, F)$, we are asked whether $H$ contains a clique of size $k$. If $w\in W_i$, then we say that $w$ has \emph{color} $i$. W.l.o.g.\ we assume that $|W_1|=|W_2|=\ldots=|W_k|=n$ and that every vertex has at least one neighbor of every color. 
    Furthermore, for all $i\in[k]$, we assume the vertices in $W_i$ are ordered in some arbitrary but fixed way, that is, $W_i=\{w^i_1,w^i_2,\ldots,w^i_n\}$.
    Let $F_{i,j}$ with $i<j$ denote the set of all edges between vertices from $W_i$ and $W_j$. We assume w.l.o.g.\ that $|F_{i,j}|=m$ for all $i< j$.
    Furthermore, for all $i<j$ we assume that the edges in $F_{i,j}$ are ordered in some arbitrary but fixed way.

    Given an instance $H$ of \textsc{Multicolored Clique}, we construct an instance $D$ of \deltaExactLong\ as follows. To ease the presentation, we first describe the underlying graph $G$ that is implicitly defined by the entries $D_{v,v'}=1$, that is, the pairs of vertices $v,v'$ that should be connected by a temporal path of duration one, meaning that there needs to an edge connecting the two vertices. Afterwards, we describe the remaining entries of $D$.

    We use the following subgraph $G_{i,j}$ as a \emph{edge selection gadget} for color combination $i,j$ with $i<j$. We first describe the vertex set of $G_{i,j}$.
    \begin{itemize}
        \item A set $X_{i,j}$ of vertices $x_1, x_2, \ldots, x_m$.
        \item A set $Y_{i,j}$ of vertices $y_1, y_2, \ldots, y_m$.
        \item A set $Z_{i,j}$ of vertices $z_1, z_2, \ldots, z_m$.
        \item Vertex sets $U_1, U_2, \ldots, U_m$ with $U_\ell=\{u^\ell_1, u^\ell_2,\ldots, u^\ell_n, u^\ell_{n+1}, u^\ell_{n+2}, u^v_{2n}\}$ for all $\ell\in[m]$. Let $U_{i,j}=\bigcup_{\ell\in[n]} U_\ell$.
        \item Two special vertices $v_{i,j}^\star,v_{i,j}^{\star\star}$.
    \end{itemize}
    Graph $G_{i,j}$ has the following edges.
    \begin{itemize}
        \item For all $\ell\in [m]$ we have edge $\{x_\ell,v_{i,j}^\star\}$, $\{y_\ell,v_{i,j}^\star\}$, and $\{z_\ell,v_{i,j}^{\star\star}\}$. 

        Furthermore, we have edges $\{y_\ell,u^\ell_1\}$ and $\{z_\ell,u^\ell_{2n}\}$.
        \item For all $\ell\in [m]$ and $\ell'\in [2n-1]$, we have edge $\{u^\ell_{\ell'},u^\ell_{\ell'+1}\}$.
    \end{itemize}
    This finishes the construction of the edge selection gadget $G_{i,j}$.

    For each color $i$, we introduce the following vertices. Intuitively, what we describe in the following will be used as a \emph{verification gadget} for color $i$.
    \begin{itemize}
        \item We have $k-1$ vertices $v^i_\ell$ for $\ell\in[k]\setminus\{i\}$ and one vertex $v^i$.
        \item If $1\neq i\neq k$, then we have a set $\hat{U}_i$ of $n$ additional vertices $\hat{u}^i_1,\hat{u}^i_2,\ldots,\hat{u}^i_n$.
    \end{itemize}
    We add the following edges. Assume $1\neq i \neq k$. We describe the two cases $i=1$ and $i=k$ afterwards. Let $e_\ell\in F_{1,i}$ let $w^i_{\ell'}\in W_i$ be incident with $e_\ell$. Then we add edge $\{v^i,u^\ell_{\ell'}\}$ between $v^i$ and the vertex $u^\ell_{\ell'}$ of the edge selection gadget $G_{1,i}$. 
    
    Let $1<j<i$ and let $e_\ell\in F_{j,i}$ let $w^i_{\ell'}\in W_i$ be incident with $e_\ell$. Then we add edge $\{v_{j-1}^i,u^\ell_{\ell'}\}$ between $v^i_{j-1}$ and the vertex $u^\ell_{\ell'}$ of the edge selection gadget $G_{j,i}$.
    Furthermore, we add edge $\{v_{j}^i,u^\ell_{\ell'+1}\}$ between $v^i_j$ and the vertex $u^\ell_{\ell'+1}$ of the edge selection gadget $G_{j,i}$.

    We add edge $\{v^i_{i-1},\hat{u}^i_1\}$ and for all $\ell\in[n-1]$ we add edge $\{\hat{u}^i_\ell,\hat{u}^i_{\ell+1}\}$. 
    
    Let $e_\ell\in F_{i,i+1}$ let $w^i_{\ell'}\in W_i$ be incident with $e_\ell$. Then we add edge $\{\hat{u}^i_n,u^\ell_{n+\ell'}\}$ between $\hat{u}^i_n$ and the vertex $u^\ell_{n+\ell'}$ of the edge selection gadget $G_{i,i+1}$.

    Let $i<j\le k$ and let $e_\ell\in F_{i,j}$ let $w^i_{\ell'}\in W_i$ be incident with $e_\ell$. Then we add edge $\{v_{j-1}^i,u^\ell_{n+\ell'}\}$ between $v^i_{j-1}$ and the vertex $u^\ell_{n+\ell'}$ of the edge selection gadget $G_{i,j}$.
    Furthermore, we add edge $\{v_{j}^i,u^\ell_{n+\ell'+1}\}$ between $v^i_j$ and the vertex $u^\ell_{n+\ell'+1}$ of the edge selection gadget $G_{i,j}$.

    Finally, we add two vertices $\hat{v},\hat{v}'$ and the following edges. Intuitively, what we describe next will be used to connect many vertex pairs by short paths, which will make arguing about possible labelings in yes-instances much easier.
    \begin{itemize}
        \item Edge $\{\hat{v},\hat{v}'\}$.
        \item An edge between $\hat{v}$ an all vertices in $X_{i,j}\cup Y_{i,j} \cup Z_{i,j}\cup U_{i,j}$ for all $1\le i<j\le k$.
        \item An edge between $\hat{v}'$ an all vertices in $X_{i,j}\cup Y_{i,j} \cup Z_{i,j}\cup U_{i,j}$ for all $1\le i<j\le k$.
    \end{itemize}
\end{proof}

\section{FPT and W[1]-hardness of exact realization -- new version}
In a graph $G=(V,E)$, a \emph{feedback edge set} $F \subset E(G)$ is a subset of edges, such that each cycle in $G$ has at least one edge in $F$.
The minimum such set $F$ is called a  \emph{minimum feedback edge set} and its size, $k = |F|$, is called the \emph{feedback edge number} of graph $G$.
Note that one can find a minimum feedback edge set in linear time, by calculating a spanning tree (or forest) $T$ of the given graph $G$ and then removing all of the edges $T$ from $G$, \ie $F = E(G) \setminus E(T)$.

\begin{theorem}\label{thm:FPTwrtFES}
    \deltaExactLong\ is FPT parameterized by the feedback edge number.
\end{theorem}

Before we start with the proof of \cref{thm:FPTwrtFES} we fix the following notation. 
Let $D$ be the input matrix of \deltaExact\ \ie
the matrix of the fastest temporal paths among $n$ vertices, and let $G$ be its underlying graph, on $n$ vertices and $m$ edges.
Let $F$ be a minimum feedback edge set of $G$ and let $k$ be the feedback edge number of $G$.

Throughout the proof, whenever we say \emph{guess} the path/subpath/vertex, etc., we mean that we enumerate all possible solutions.
For example, let $u$ and $v$ be two vertices in a graph $G$. 
We say that we \emph{guess} that $P$ is the underlying path of a fastest temporal path $(P, \lambda)$ from $u$ to $v$ in $(G, \lambda)$.
This means that we iterate over all possible possible $u,v$-paths $P_1, P_2, \dots $ in $G$, 
and form multiple cases, 
one in which $(P_1, \lambda)$ is a fastest temporal path from $u$ to $v$ in $(G,\lambda)$,
one when $(P_2, \lambda)$ is a fastest temporal path from $u$ to $v$ in $(G,\lambda)$, etc.
To correctly determine if the problem has a valid solution, we must check all cases.
In each case we perform required calculations, 
and we either come to the contradiction (meaning that the current guess was wrong) 
or our assumption holds (meaning that the guessed path is really a fastest path from $u$ to $v$ in $(G, \lambda)$).

Note that we can easily solve the \deltaExact\ by guessing the structure of temporal paths among all pairs of vertices.
But, this would create $f(n,m)$ guesses (where $f(n,m)$ is an exponential function), and the running time of the algorithm would be too big.
Therefore, we approach the problem using some special properties, which allow us to create $f(k)$ different guesses.
We solve each guess by creating an ILP instance with $O(k^3)$ variables and $O(n^c)$ constraints, for some $c \in \mathbb{N}$.
The ILP instance is a \textsc{Yes} instance if and only if $D$ is a \textsc{Yes} instance of \deltaExact.
Moreover, from a positive solution of the ILP instance we can construct the desired labeling $\lambda$.
Using results from \cite{Lenstra1983Integer} we know that an ILP instance can be solved in FPT time with respect to the number of variables.
Therefore we can solve each of our ILP instances in FPT time with respect to $k$.

We now introduce the algorithm, that determines the labeling of $G$, satisfying $D$, and runs in $f(k) \cdot \text{poly}(n)$ time.
Our algorithm performs the following steps.
\begin{enumerate}
    \item Preprocessing on graph $G$. Determine the set $E$ of feedback edges, the set $U$ of vertices of interest and the set $U^*$ of neighbors of vertices of interest, where $|E|, |U|, |U^*| \in O(k)$.
    \item Guess what are the fastest temporal paths among vertices in $U$ and 
    what are the fastest temporal paths among the vertices in $U^*$, that are of some specific form.
    \begin{enumerate}
        \item For each guess create an ILP instance $I$ with $f(k)$ variables.
        \item Add constraints for all pairs of vertices.
    \end{enumerate}
    \item Solve the ILP instance $I$ and check if a solution exists. In the positive case create its corresponding labeling $\lambda_I$ and check if it satisfies $D$.
    All in total we create $f(k)$ partial ILP instances, each with $f(k)$ variables and $O(n^c)$ constraints.
\end{enumerate}

\subsection{Preprocessing of the input}
From the underlying graph $G$ of $D$ we first create a graph $G'$ by
iteratively removing vertices of degree one from $G$.
Then we determine a minimum feedback edge set $F$ of $G'$ by finding a spanning tree $T$ of $G'$ and set $F = E(G) \setminus E(T)$. 
Note that $F$ is also a minimum feedback edge set of the whole $G$.
Lastly, we determine sets $U$, of \emph{vertices of interest}, and $U^*$ of the neighbors of vertices of interest, in the following way.
Let $T$ be a spanning tree of $G'$, with $F$ being the corresponding feedback edge set of $G'$.
Let $V_1 \subset V(G')$ be the set of leaves in the spanning tree $T$, $V_2 \subset V(G')$ be the set of vertices of degree two in $T$, that are incident to at least one edge in $F$, 
and let $V_3 \subset V(G')$ be the set of vertices of degree at least $3$ in $T$. 
Then $|V_1| + |V_2| \leq 2k$, since every leaf in $T$ and every vertex in $V_2$ is incident to at least one edge in $F$,
and $|V_3| \leq |V_1|$ by the properties of trees.
We denote with $U = V_1 \cup V_2 \cup V_3$ the set of vertices of interest. It follows that $|U| \leq 4k$.
We set $U^*$ to be the set of vertices in $V(G') \setminus U$ that are neighbors of vertices in $U$, \ie $U^* = \{v \in V(G') \setminus U | u \in U, v \in N(u)\}$.
Again, using the tree structure, we get that for any $u \in U$ its neighborhood is of size $|N(u)| \in O(k)$, therefore $U^* \in O(k^2)$.

The whole Step 1 is performed in $O(n + m)$ time.


Let us denote with $Z$ the set of vertices removed from $G$ at the beginning of the algorithm, \ie $Z = V(G) \setminus V(G')$.
By the construction it follows that $Z$ consists of disjoint trees. 
Note that the fastest temporal paths among vertices of $G'$ are independent of vertices in $Z$.
Let us remember the following basic property of the problem \deltaExact.
A labeling $\lambda$ of $G$ satisfies $D$ if every fastest temporal path from vertex $v_i$ to $v_j$ equals $D_{v_i, v_j}$,
and none other temporal path from $v_i$ to $v_j$ is faster.
In order to find a labeling that satisfies this property we split our analysis in nine cases,
where the first vertex is in one of the sets $U,V(G') \setminus U, Z$,
and similarly the last vertex is in one of the sets $U,V(G') \setminus U, Z$.
In each of these cases we guess the underlying path $P$ of the fastest temporal path from the vertex $v_i$ to $v_j$, 
which results in one equality constraint for the labels on the path $P$. 
For all other temporal paths from $v_i$ to $v_j$ we know that they cannot be faster, so we introduce inequality constraints for them.
This results in producing $poly(n,m)$ constraints. We have to guarantee only, that the number of guesses, of faster temporal paths, is exponential in $k$ and not in $n$.

For an easier understanding and the analysis of the algorithm we give the following definition.
\begin{definition}
    Let $U \subseteq V(G')$ be a set of vertices of interest and let $u,v \in U$.
    A path $P = (u=v_1,v_2, \dots, v_p = v)$  in graph $G'$, where all inner vertices are not in $U$, \ie $v_i \notin U$ for $i \in \{ 2, 3, \dots, p-1\}$,
    is called a \emph{segment} from $u$ to $v$. We denote it as $S_{u,v}$.
    Note $S_{u,v} \neq S_{v,u}$.
\end{definition}
From the definition it follows, that there are $O(k^2)$ segments in $G'$.
Observe also that a temporal path in $G'$ between two vertices of interest is either a segment, or consists of a sequence of some segments.
%For a pair $u,v$ of vertices in $U$, there are $O(k!)=O(k^k)$ possible paths in $G'$ between them.  

\subsection{Creating ILP instances and determining variables}
We start the algorithm by guessing necessary structures.
Note, whenever we say that we guess the fastest temporal path between two vertices, we mean that we guess the underlying path of the fastest temporal path among those two vertices.
In the case when we want to guess the fastest path from $u$, that (somehow) passes through the vertex $x$, and goes directly to $v$ via an edge, we write it as a fastest path of form $u \leadsto x \rightarrow v$.
Similarly it holds if we want to include more vertices, that has to be visited by the path. If there is an edge (\ie a unique path) among two vertices, we denote it by $\rightarrow$,
if the path among two vertices is not uniquely determined, we denote it by $\leadsto$.
We guess the following structures.
\begin{enumerate}
    \item The fastest temporal paths among all pairs of vertices of $U$.
    For a pair $u,v$ of vertices in $U$, there are $k!$ possible paths in $G'$ between them. 
    Therefore, we have to try all $O(k^k)$ possible paths, where at least one of them will be a fastest temporal path from $u$ to $v$, respecting the values from $D$.
    %If this does not happen we conclude that $D$ is a \textsc{No} instance of \deltaExact.
    Repeating this procedure for all pairs of vertices $u,v \in U$ we get $O((k^k)^{k^2})=O(k^{k^3})$ different variations of the fastest temporal paths among all vertices in $U$.
    \item The fastest temporal paths among all vertices in $U^*$, 
    which by similar arguing as for vertices in $U$, gives us $O(k^{k^6})$ guesses.
    \item The fastest temporal paths from vertices of $U$ to vertices in $U^*$,
    and vice versa, the fastest temporal paths from vertices in $U^*$ to vertices in $U$.
    This gives us $O(k^{k^4})$ guesses.
    \item Let $S_{u,v} = (u=v_1,v_2, \dots, v_p = v)$ and $S_{w,z} = (w=z_1,z_2, \dots, z_r = z)$ be two segments,
    first one between vertices $u,v \in U$, second one between vertices $w, z \in U$.
    We want to guess the following fastest temporal paths
    $v_2 \rightarrow u \leadsto w \rightarrow z_2$,
    $v_2 \rightarrow u \leadsto z \rightarrow z_2$,
    $v_2 \rightarrow u \leadsto w \rightarrow z_{r-1}$ and
    $v_2 \rightarrow u \leadsto z \rightarrow z_{r-1}$.
    We repeat this procedure for all pairs of segments.
    Since there are $O(k^2)$ segments in $G'$,
    there are $O(k^{k^5})$ possible paths of this form.
    \item \label{FPT:guess-uToSegment}
    For a segment $S_{u,v} = (u=v_1,v_2, \dots, v_p = v)$
    and a vertex of interest $w \in U$,
    we guess the following fastest temporal paths
    $w \leadsto u \rightarrow v_2$, $w \leadsto v \rightarrow v_2$,
    and
    $v_2 \rightarrow u \leadsto w$, $v_2 \rightarrow v \leadsto w$.
    \\
    For a fixed segment $S_{u,v}$ and a fixed vertex of interest $w$ we have $O(k^k)$ different possible such paths, therefore we make $O(k^{k^4})$ guesses for these paths.
    %All together we made $O(k^{k^4}$  new guesses for the structure of all of the above mentioned fastest temporal paths.
%Each of the previous partial ILP instances is therefore extended into $O(k^6)$ new ILP instances.
    \item \label{FPT:guess-splitFromAnotherSegmentAndPaths}
    Let $S_{u,v} = (u=v_1,v_2, \dots, v_p = v)$ be a line segment in $G'$, and let us
    fix a vertex $v_i \in S_{u,v} \setminus \{u,v\}$.
    Let 
    $S_{w,z} = (w=z_1,z_2, \dots, z_r = z)$ be another segment in $G'$.
    Note that all inner vertices of $S_{w,z}$ (\ie vertices in $S_{w,z} \setminus \{w,z\}$)
    can be reached from $v_i$ by some path that has to go through $w$ oz $z$.
    Therefore, the values of the duration of the fastest path from $v_i$ to vertices on 
    the segment
    $S_{w,z}$ have to increase up to a certain point, when traversing the segment coming through $w$ and coming through $z$.
    We split the analysis into two cases.
    \begin{enumerate}
        \item 
    There is a single vertex $z_j \in S_{w,z}$ for which the duration from $v_i$ is the biggest.
    More specifically, $z_j \in S_{w,z} \setminus \{w,z\}$ is the vertex with the biggest value  $D_{v_i,z_j}$.
    We call this vertex a \emph{split vertex of $v_i$ in the segment $S_{wz}$}.
    Then it holds that $D_{v_i,z_2} < D_{v_i,z_3} < \dots < D_{v_i,z_j}$ and 
    $D_{v_i,z_{p-1}} < D_{v_i,z_{p-2}} < \dots < D_{v_i,z_j}$.
    From this it follows that the fastest temporal paths from $v_i$ to $z_2, z_3, \dots, z_{j-1}$ go through $w$,
    and 
    the fastest temporal paths from $v_i$ to $z_{p-1}, z_{p-2}, \dots, z_{j+1}$ go through $z$.
    We now want to guess which vertex $w$ or $z$ is on a fastest temporal path from $v_i$ to $z_j$.
    Similarly,
    all fastest temporal paths starting at $v_i$ have to go either through $u$ or through $v$,
    which also gives us two extra guesses for the fastest temporal path from $v_i$ to $z_j$.
    Therefore, all together we have $4$ possibilities on how the fastest temporal path from $v_i$ to $z_j$ starts and ends.
    Besides that we want to guess also hot the fastest temporal paths from $v_i$ to $z_{j-1}, z_{j+1}$ start and end.
    Note that one of these is the subpath of the fastest temporal path from $v_i$ to $z_j$, and the ending part is uniquely determined for both of them,
    \ie to reach $z_{j-1}$ the fastest temporal path travels through $w$, and to reach $z_{j+1}$ the fastest temporal path travels through $z$.
    Therefore we have to determine only how the path starts, namely if it travels through $u$ or $v$.
    This introduces two extra guesses.
    For a fixed $S_{u,v}, v_i$ and $S_{w,z}$ we find the vertex $z_j$ in polynomial time, 
    or determine that $z_j$ does not exist.
    We then make four guesses where we determine how the fastest temporal path from $v_i$ to $z_j$ passes through vertices $u,v$ and $w,z$ and 
    for each of them two extra guesses to determine the fastest temporal path from $v_i$ to $z_{j-1}$ and from $v_i$ to $z_{j+1}$.
    We repeat this procedure for all pairs of segments,
    which results in producing $O(k^{k^6})$ new guesses.
    Note, $v_i \in S_{u,v}$ is fixed when calculating the split vertex for all other segments $S_{w,z}$.
    \item 
    There are two vertices $z_j, z_{j+1} \in S_{w,z}$ for which the duration from $v_i$ is the biggest.
    More specifically, $z_j, z_{j+1} \in S_{w,z} \setminus \{w,z\}$ are the vertices with the biggest value  $D_{v_i,z_j} = D_{v_i,z_{j+1}}$.
    Then it holds that $D_{v_i,z_2} < D_{v_i,z_3} < \dots < D_{v_i,z_j} = D_{v_i,z_{j+1}} > D_{v_i,z_{j + 2}} > \cdots > D_{v_i,z_{p-1}}$ and 
    $D_{v_i,z_{p-1}} < D_{v_i,z_{p-2}} < \dots < D_{v_i,z_j}$.
    From this it follows that the fastest temporal paths from $v_i$ to $z_2, z_3, \dots, z_{j}$ go through $w$,
    and 
    the fastest temporal paths from $v_i$ to $z_{p-1}, z_{p-2}, \dots, z_{j+1}$ go through $z$.
    In this case we only need to guess the following two fastest temporal paths $u \leadsto w \rightarrow z_2$
    and $u \leadsto z \rightarrow z_{p-1}$.
    Each of this paths we then uniquely extend along the segment $S_{w,z}$ up to the vertex $v_j$, resp.~$v_{j+1}$,
    which give us fastest temporal paths from $u$ to $v_j$ and from $u$ to $v_{j+1}$.
    Note that in this case we do not introduce any new guesses, as we guessed these paths already in Step~\ref{FPT:guess-uToSegment}.
    \end{enumerate}
\end{enumerate}
We create all of the guesses independently.
We start with the first one, that results in $O(k^{k^3})$ different possibilities, then we split each one of these guesses into $O(k^{k^6})$ new ones, that respond to the guessing in the second step, etc.
After creating all of the guesses we end up with exponential in $k$ different cases.
We now create one partial ILP instance for each case.
From now on we assume that we know exactly the structure of all paths we have guessed and the permutation of all variables.

\subsection{Properties of 
\texorpdfstring{\deltaExact } {Simple Delta-TGR}
}

In this section we study the properties of our problem, that then help us creating constraints of our ILP.
In our analysis we exploit the following greatly, that is why we state is as an observation.

\begin{observation}\label{obs:FirstLabelAndDuration}
    Let $P$ be the underlying path of a fastest temporal path from $u$ to $v$, where $e_1, e_p \in P$ are its first and last edge, respectively.
    Then, knowing the label $\lambda (e_1)$ of the first edge and the duration $d(P,\lambda)$ of the temporal path $(P,\lambda)$, we can uniquely determine the label $\lambda (e_p)$ of the last edge of $P$.
    Symmetrically, knowing $\lambda(e_p)$ and $d(P)$, we can uniquely determine $\lambda(e_1)$.
\end{observation}
The correctness of the above statement follows directly from \cref{def:temporalPath+Duration}. Since the duration of $(P,\lambda)$ is calculated as the difference of labels of last and first edge plus $1$,
where the label of last edge is considered with some delta periods,
\ie $d(P,\lambda) = p_i \Delta + \lambda(e_p) - \lambda (e_1) + 1$, for some $p_i \geq 0$.
Therefore $d(P,\lambda) \pmod \Delta \equiv  (\lambda(e_p) - \lambda (e_1) + 1) \pmod \Delta$.

In the following we prove that knowing the structure (the underlying path) of a fastest temporal path $P$, from a vertex of interest $u$ to a vertex of interest $v$,
results in determining the labeling of each edge in the fastest temporal path from $u$ to $v$ 
(with the exception of some constant number of edges), with respect to the label of the first edge.
More precisely, if path $P$ from $u$ to $v$ is a segment, then we can determine labels of all edges,
and if $P$ consists of $\ell$ segments, then we can determine the labels of all but $\ell -1$ edges.
For the exact formulation and proofs see \cref{claim:FPT-uv-Labelalledges,claim:FPT-uv-LabelAlmostalledges}.

\begin{claim}\label{claim:FPT-uv-Labelalledges}
    Let $u, v \in U$ be two arbitrary vertices of interest and suppose that $P = (u=v_1,v_2, \dots, v_p = v)$, where $p \geq 2$, 
    is a $u,v$ path in $G'$, which is also the underlying path of a fastest temporal path from $u$ to $v$.
    Moreover suppose also that $P$ is a segment.
    %Suppose also that no other vertex from $P$ is a vertex of interest, \ie $P \setminus \{u,v\} \cap U = \emptyset$.
    We can determine the labeling $\lambda$ of every edge in $P$ with respect to the label $\lambda(uv_2)$ of the first edge.
    %such that the labeling $\lambda$ respects $D'$.
\end{claim}

\begin{proof}
We claim that $u$ reaches all of the vertices in $P$ the fastest, when traveling along $P$ (\ie by using a subpath of $P$).
To prove this suppose for the contradiction that there is a vertex $v_i \in P \setminus \{u,v\}$, that is reached from $v$ on a path different than $P_i = (u, v_2, v_3, \dots, v_i)$.
Since the only vertices of interest of $P$ are $u$ and $v$, it follows that all other vertices on $P$ are of degree $2$. 
Then the only way to reach $v_i$ from $u$, that differs from $P$, would be to go from $u$ to $v$ using a different path $P_2$,
and then go from $v$ to $v_{p-1}, v_{p-2}, \dots, v_i$.
But since $P$ is the fastest temporal path from $u$ to $v$, we get that $d(P_2) \geq d(P)$ and $d(P_2 + (v,v_{p-1}, \dots, v_i) > d(P) > d(P_i)$.

Now to label $P$ we use the fact that the fastest temporal path from $u$ to any $v_i \in P$ is a subpath of $P$, 
therefore we can label each edge using \cref{obs:FirstLabelAndDuration},
where the duration from $u$ to $v_i$ equals to $D_{u,v_i}$ and 
we set the first label of $P$ to be a constant $c_{uv}$.
This gives us a unique label for each edge of $P$, that depends on the value $c_{u,v}$.
\end{proof}

\begin{claim}\label{claim:FPT-uv-LabelAlmostalledges}
    Let $u, v \in U$ be two arbitrary vertices of interest and suppose that $P = (u=v_1, v_2, \dots, v_p = v)$, where $p \geq 2$, 
    is a $u,v$ path in $G'$, which is also the underlying path of a fastest temporal path from $u$ to $v$.
    Let $\ell_{u,v} \geq 1$ be the number of vertices of interest in $P$ different to $u,v$, namely $\ell_{u,v} = | v_i \in \{P \setminus \{u,v \} \} \cap U |$.
    We can determine the labeling $\lambda$ of all but $\ell_{u,v}$ edges of $P$, with respect to the label $\lambda(uv_2)$ of the first edge,
    such that the labeling $\lambda$ respects the values from $D$.
    Moreover, for the labels of edges that cannot be determined precisely, we can get some extra restrictions.
\end{claim}

\begin{proof}
We denote with $v_i \in U$ a vertex of interest in $P \setminus \{u,v\}$.
There are two options, either $v_i$ is reached from $u$ using the subpath $P_i = (u, v_2, v_3, \dots, v_i)$ of $P$,
or there exists a fastest temporal path $P' = (u, w_2, w_3, \dots, w_{p'} = v_i)$, for which $P \cap P' = \{u, v_i\}$.
If the temporal path from $u$ to $v_i$ is a subpath of $P$, then we determine the labeling of $P$ using \cref{claim:FPT-uv-Labelalledges}.
So we can suppose that the fastest temporal path from $u$ to $v_i$ is of form $P'$.\todo{TBD - George proved it}
\end{proof}

%Observe also the following.

\begin{claim}\label{claim:FPT-unlabeldPaths-01}
    Suppose that
    $S_{u,v}, S_{w,z}$ are two segments with $ v_i \in S_{u,v}$ and $z_j \in S_{w,z}$,
    where $z_j$ is a split vertex of $v_i$ in the segment $S_{w,z}$.
    W.l.o.g. suppose that the fastest temporal path from $v_i$ to $z_j$ travels through vertices $u$ and $w$.
    Then the fastest temporal path from $v_i$ to any other vertex of $S_{w,z}$, that is closer to $w$,
    travels through the same two vertices $u$ and $w$.
    Similarly it holds for the cases when the fastest temporal path travels through $w,v$ or $z,u$ or $z,v$.
\end{claim}
\begin{proof}
    Let $z_\ell$ be a vertex of $S_{w,z}$, that is closer to $w$ than $z$ in the segment.
    Let us denote with $P_{v_i,z_j}$ the underlying path of the fastest temporal path from $v_i$ to $z_j$.
    Denote with $P_{v_i,z_j}^\ell$
    the subpath of the fastest temporal path from $v_i$ to $z_j$, that terminates in $z_\ell$.
    We want to show that $P_{v_i,z_j}^\ell$ is an underlying path of a fastest temporal path from $v_i$ to $z_j$.
    Let us observe the following possibilities.
    \\
    First, suppose for the contradiction, that the fastest temporal path from $v_i$ to $z_\ell$ travels through vertices $u$ and $z$.
    Denote this path as $P^1_{v_i,z_\ell}$.
    Then it follows that $d(P^1_{v_i,z_\ell}, \lambda) \leq d(P_{v_i,z_j}^\ell,\lambda)$, which would imply that
    the duration of the temporal path from $v_i$ to $z_j$ using the subpath of $P^1_{v_i,z_\ell}$, would be strictly smaller than the duration of $(P_{v_i,z_j},\lambda)$, which cannot be possible.
    \\
    Second, suppose that the fastest temporal path from $v_i$ to $z_\ell$ travels through vertices $v$ and $w$.
    Denote this path as $P^2_{v_i,z_\ell}$.
    Note that $P_{v_i,z_j}^\ell$ and $P^2_{v_i,z_\ell}$ intersect on a segment $S_{w,z}$ from the vertex $w$ to $z_\ell$.
    Therefore since
    $d(P^2_{v_i,z_\ell}, \lambda) \leq d(P_{v_i,z_j}^\ell,\lambda)$,
    and since there is unique way to
    extend the path $P^2_{v_i,z_\ell}$ from $z_\ell$ to $z_j$, denote the extended path as $P^j_{v_i,z_\ell}$,
    we get that $d(P^j_{v_i,z_\ell}, \lambda) \leq $$(P_{v_i,z_j},\lambda)$.
    Which implies that $d(P^j_{v_i,z_\ell}, \lambda) = d(P_{v_i,z_j}, \lambda)$.
    Now using the similar argument it follows that $d(P_{v_i,z_j}^\ell, \lambda) = d(P^2_{v_i,z_\ell}, \lambda)$,
    therefore $P_{v_i,z_j}^\ell$ is also a fastest temporal path from $v_i$ to $z_j$.
    \\
    Third, suppose that the fastest temporal path from $v_i$ to $z_\ell$ travels through vertices $v$ and $z$.
    Denote this path as $P^3_{v_i,z_\ell}$.
    Then the duration of the temporal path from $v_i$ to $z_j$ using the subpath of $P^3_{v_i,z_\ell}$, would be strictly smaller than the duration of $(P_{v_i,z_j},\lambda)$, which cannot be possible.
\end{proof}


\subsection{Adding constraints and variables to the ILP}
First case we observe is the case when we want to determine the labels on fastest temporal paths among vertices of interest.
We proceed in the following way.
Let $u,v \in U$ be two vertices of interest and let $P_{u,v}$ be the fastest temporal path from $u$ to $v$.
If $P_{u,v}$ is a segment we determine all the labels of edges of $P$, with respect to the label of the first edge (see \cref{claim:FPT-uv-Labelalledges}).
%Note that since we determined labels of all edges, with respect to the first edge, we also know the exact values of waiting times at each inner vertex of $P_{u,v}$, when traversing the path.
In the case when $P{_u,v}$ is a sequence of $\ell$ segments, we determine all but $\ell - 1$ labels of edges of $P$, with respect to the label of the first edge (see \cref{claim:FPT-uv-LabelAlmostalledges}).
%In both cases we introduce the equality constraint for $P_{u,v}$ (as in \cref{eq:FPT-equalityConstraint}).
%Since there are $\ell - 1$ edge labels that are not determined with respect to the label of the first edge, may be some  TODO from above??

After finishing the previous step for all pairs of vertices in $U$,
the edges of fastest temporal paths from $u$ to $v$, where $u,v \in U$, are determined with respect to the label of the first edge of each path.
Let us call these edges \emph{determined edges}, and the edges that were not determined with the respect to some other edge label up to now the \emph{undetermined} edges.
If the fastest temporal path among two vertices $u,v \in U$ is just an edge $e$, then we treat it as being determined, since it gets assigned a label $\lambda(e)$ with respect to itself.
Note that the undetermined edges are exactly the ones that are not a part of any fastest temporal path.
Let us first observe some properties of such edges.

\begin{claim}
    Undetermined edges form disjoint segments in $G'$, where each segment starts and ends in some vertex of interest, 
    \ie for every undetermined path $P$ there exist $u,v \in U$, such that $P = (u, v_1, v_2, \dots, v_p, v)$, where $p \geq 0$ and $v_1,v_2,\dots,v_p \in V(G') \setminus U$.
\end{claim}
\begin{proof}
    We prove this claim in two steps.
    First we prove that all inner vertices of $P$, \ie vertices $v_i \in  S \setminus \{u,v\}$ are of degree exactly $2$ and are not vertices of interest. 
    They are of degree at least $2$ since they are on a path $P$ and are incident to two undetermined edges in  $P$.
    They are of degree at most $2$ as if they were of a higher degree they would be vertices of interest and in this case we could split $P$ into shorter segments (subpaths).
    
    Second, suppose that $P$ does not start in a vertex of interest, therefore $u \notin U$. As $u$ is the starting vertex of the undetermined path $P$, there must exist a vertex of interest $u'$ such that a path between $u$ and $u'$ is labeled.
    Since $u$ is not a vertex of interest, the only reason to label edges between $u'$ and $u$ would be if they were a part of some fastest temporal path among two vertices of interest and $u$ would be an inner vertex on that path.
    Therefore $u$ would have to be incident to two labeled edges (that are part of some fastest temporal path) and one undetermined edge, but in this case the degree of $u$ would be at least $3$ and therefore $u$ would be a vertex of interest.

    From all of the above, it follows that $P$ is an undetermined segment in $G'$.
\end{proof}

\begin{claim}
    After determining the labels of fastest temporal paths between all pairs of vertices of interest,
    there can be at most one undetermined segment among each pair of them.
    %among any pair of vertices of interest $u,v \in U$ there can be at most one undetermined segment $S = (u, v_1, v_2, \dots, v_p, v)$, where $p \geq 0$ and $v_1,v_2,\dots,v_p \in V(G') \setminus U$.
\end{claim}
\begin{proof}
   Suppose for the contradiction that after determining the labels of all fastest temporal paths between vertices of interest,
   there are two undetermined segments among $u,v \in U$.
   Denote them with $S_{u,v}$ and $S'_{u,v}$. 
   Then $C = S_{u,v} \cup S_{u,v}'$ forms a cycle in $G'$. Since $G'$ admits no loops, the cycle $C$ is of length at least $3$.
   Since $C$ is a cycle, at least one of its edges $e \in E(C)$ has to be in the feedback edge set $F$, and therefore, by the definition of vertices of interest $U$, 
   both endpoints of $e$ have to be in $U$.
   If $e = uv$ then there are only two vertices of interest in $C$, but the label of edge $e$ is determined.
   And if $e \neq uv$ then there are at least $3$ vertices of interest in $C$.
   So $S_{u,v}$ and $S_{u,v}'$ cannot be both fully undetermined.
\end{proof}

%From the above it follows that there are at most $O(|U|^2) = O(k^2)$ undetermined segments in $G'$ after the algorithm finishes with determining the labels for all fastest temporal paths among vertices of interest.
Now we want to relate the undetermined segments with the determined ones.
Let $S_{u,v}$ and $S_{w,z}$ be two segments.
At the beginning we have guessed what is a split vertex of a fixed vertex $v_i \in S_{u,v}$ in the segment  $S_{w,z}$. 
Denote this vertex as $z_j$, denote its neighbor closer to $w$ in $S_{w,z}$ as $z_{j-1}$,
and its neighbor closer to $z$ in $S_{w,z}$ as $z_{j+1}$.
We also guessed how the fastest temporal path from $v_i$ to $z_j$ starts and ends, which determined also the structure of the fastest temporal paths 
from $v_i$ to $z_{j-1}$ and from $v_i$ to $z_{j+1}$
(for the details see \cref{FPT:guess-splitFromAnotherSegmentAndPaths}).
W.l.o.g. assume that we have guessed that the fastest path from $v_i$ to $z_j$
passes through $w$ and $z_{j-1}$.
Then the fastest temporal path from $v_i$ to $z_{j+1}$ passes through $z$.
And all fastest temporal paths from $v_i$ to any $z_{j'} \in S_{w,z}$
use all of the edges in $S_{w,z}$ with the exception of the edge $z_j z_{j+1}$.
Using this information and \cref{obs:FirstLabelAndDuration}, we can determine the labels on all edges, with respect to the first or last label from the segment $S_{u,v}$,
with the exception of the edge $z_j z_{j+1}$.
Therefore all edges of $S_{w,z}$ but $z_j z_{j+1}$ become determined.
Since we repeat that procedure for all pairs of segments,
we get that for a fixed segment $S_{w,z}$ we end up with an undetermined edge $z_j z_{j+1}$
if and only if, this is an undetermined edge in relation to every other segment $S_{u,v}$ and its fixed vertex $v_i$.

We repeat this procedure for all pairs of segments.
Each specific calculation takes linear time, since there are $O(k^2)$ segments, this calculation takes $O(k^4)$ time.
At this point
the edges of every segment are fully determined, 
with the exception of at most two edges per segment (the first one and potentially one extra somewhere in the segment).
For every edge $e$ in $G'$ that is incident to a vertex of interest we create a variable $x_e$ that can have values from $ \{1, 2, \dots, \Delta\}$.
Besides that we create one variable for each edge that is still undetermined on a segment.
Since each vertex of interest is incident to at most $k$ edges, and each segment has at most one extra undetermined edge, we create $O(k^2)$ variables.
At the end we guess the permutation of all $O(k^2)$ variables,
together with the relation of each variable to the labels of edges incident to these undetermined edges.
Namely, for an edge $e$ that is undetermined, we set its value to $x_e$ and check labels of all of its neighbors,
which are determined by some other label,
and variables of the undetermined neighbors,
and guess if $x_e$ is smaller, equal or bigger than the labels of the edges of its neighbors.
So, for any two variables $x_e$ and $x_f$, we know if $x_e < x_f$ or $x_e = x_f$, or $x_e > x_f$,
and for any neighboring edge $g$ of $e$ we know if $x_e < \lambda(g)$ or 
$x_e = \lambda(g)$, or $x_e > \lambda(g)$.
This results in $O(k^2 !) = O(k^{2k^2}) = O(k^{k^3})$ guesses
and consequently
each of the ILP instances we created up to now is further split into $O(k^{k^3})$ new ones.

We proceed with adding constraints to each of our ILP instances, by iterating through all pairs of vertices.
We now describe how we add constraints to a path. Whenever we say that a duration of a path gives an equality or inequality constraint, we mean the following.
Let $P=(u=v_1,v_2, \dots, v_p = v)$ be the underlying path of a fastest temporal path from $u$ to $v$,
and let $Q = (u=z_1,z_2, \dots, z_r = v)$ be the underlying path of another temporal path from $u$ to $v$.
Then we know that $d(P,\lambda) = D_{u,v}$ and $d(Q, \lambda) \geq D_{u,v}$.
Using \cref{obs:durationPwithWaitingTimes}
we create an \emph{equality constraint} for $P$
of the form 
\begin{equation}\label{eq:FPT-equalityConstraint}
    D_{u,v} = \sum _ {i=2} ^ {p-1} (\lambda (v_{i}v_{i+1}) - \lambda (v_{i-1}v_i))_\Delta + 1,
\end{equation}
and an \emph{inequality constraint} for $Q$ 
\begin{equation}\label{eq:FPT-inequalityConstraint}
    D_{u,v} \leq \sum _ {i=2} ^ {r-1} (\lambda (z_{i}z_{i+1}) - \lambda (z_{i-1}z_i))_\Delta + 1.
\end{equation}
In both cases we implicitly assume that if the difference of $(\lambda (z_{i}z_{i+1}) - \lambda (z_{i-1}z_i))$ is smaller than zero, for some $i$, we add the value $\Delta$ to it (\ie we consider the difference modulo $\Delta$), therefore we have the sign $\Delta$ around the brackets.
Note that we know if the difference of two consecutive labels is positive or negative. 
In the case when two consecutive labels are determined with respect to the same label $\lambda(e)$ the difference among them is easy to determine,
if one or both consecutive labels are undetermined 
then we have guessed in what kind of relation they are,
the case when two consecutive labels are determined with respect to different labels $\lambda(e)$ and $\lambda(f)$.
Therefore we know when $\Delta$ has to be added, which implies that \cref{eq:FPT-equalityConstraint,eq:FPT-inequalityConstraint} are calculated correctly for all paths.

We iterate through all pairs of vertices $x,y$ and make sure that the fastest temporal path from $x$ to $y$ produces the equality constrain \cref{eq:FPT-equalityConstraint},
and all other temporal paths from $x$ to $y$ produce the inequality constraint \cref{eq:FPT-inequalityConstraint}.

First we consider the case when $u,v$ are two vertices of interest.
This was the case that we partially studied at the beginning of the algorithm, that helped us determine certain labels of the graph.
We now iterate again through all fastest temporal paths from $u$ to $v$ and introduce an equality constraint for them.
We then continue 
through all of the paths from $u$ to $v$, and for every one, that is not the underlying path of a fastest temporal path, we add the inequality constraint.
There are $O(k^k)$ possible paths from $u$ to $v$ in $G'$, therefore in this step we introduce $O(k^{k^3})$ constraints.

Next we continue with the case of the fastest path from $u$ to $x$, where $u \in U$ and $x \in V(G') \setminus U$.
We have two possibilities:
\begin{enumerate}[(i)]
    \item Vertex $x$ is on a fastest temporal path from $u$ to some other vertex of interest $v$.
    Suppose that $x$ lies on a segment $S_{w,z}$.
    Then by \cref{claim:FPT-uv-LabelAlmostalledges} we know that the fastest temporal path from $u$ to $x$ is a subpath of the fastest temporal path from $u$ to $v$,
    which hast to be of the duration $D_{u,x}$. This results in adding an equality constraint to our ILP.
    We now have to iterate through all others temporal paths from $u$ to $x$ and add the constraint to ILP that states that their duration is not smaller than the value $D_{u,x}$.
    To reach $x$ from $u$, we have to first travel from $u$ to $w$ (resp.~$z$), and then travel through $S_{u,w}$ to $x$.
    There are $O(k^k)$ different paths to reach $w$ (resp.~$z$), which are then uniquely extended to $x$.
    Therefore we add $O(k^k)$ inequality constraints for all these paths.
    
    \item \label{item:FPT-uToxcase2}
    Vertex $x$ is not on a fastest temporal paths from $u$ to any $v \in U$.
    Suppose that $x$ lies on a segment $S_{w,z} = (w,z_1,z_3, \dots, z_p, z)$.
    Since we know the underlying path of the fastest temporal path from $u$ to $z_1$ through $w$ and from $u$ to $z_p$ through $z$,
    we have to only determine how we reach vertex $x$ from the segment (\ie will the fastest path go through $w$ or through $z$).
    This is done,
    by checking the distances from $u$ to $z_i$, for $z_i \in S_{w,z} \setminus \{w,z\}$,
    and determining the split vertex $w^*$ from $u$ on the segment $S_{w,z}$.
    If $x \neq w^*$, we know which part of the segment is used to reach $x$.
    We set this path as the fastest temporal path, that must satisfy the equalities and for all other temporal paths
    (\ie paths from $u$ to $z$ and then to $x$, and paths from $u$ to $w$ and then to $x$),
    we add the inequalities constraints.
    If $x = w^*$, we have two possibilities, we check the values of each one and set the minimum of them, to equal the $D_{u,x}$, 
    if this is even possible, if one is too small we immediately reject.
    Note, we know the durations of fastest temporal paths from $u$ to $z_1$ and $z_p$. We also know all the labels of the edges of $S_{w,z}$, with respect to each other,
    with the exception of at most one inner edge.
    Therefore we can completely calculate the duration from $u$ to $x=w^*$ coming from one side (\ie either from $w$ or $z$).
    Suppose that the value of the duration of the fastest path from $u$ to $w^*$, through $w$ is
    smaller than $D_{u,x}$, then we reject and stop the calculation.
    If the value equals $D_{u,x}$ we have determined the fastest temporal path and we use it to create equality constraints, while we use the other path, through $z$, to create an inequality constraint.
    If the value is bigger than $D_{u,x}$, we create an inequality constraint for this path and create an equality constraint for the path through $z$.
    We calculate also all other possible paths from $u$ to $x$ and create inequalities for all of them.
    These paths are exactly all the paths from $u$ to $w$ (resp.~$z$), from where we uniquely continue to $x$ through $S_{w,z}$.
    There are $O(k^{k})$ different paths from $u$ to $w$ (resp.~$z$), therefore we add
    $O(k^k)$ inequality constraints for all these paths, for a fixed $u$ and $x$.   
\end{enumerate} 

Now we proceed with the case of determining the fastest path from $x$ to $u$, where $x \in V(G') \setminus U$ and $u \in U$.
%We use the similar approach as in the case~\ref{item:FPT-uToxcase2} from above.
Let $x$ a vertex in the segment $S_{w,z} = (w,z_1,z_2, \dots, z_p, z)$.
We can easily determine the vertex $z_i \in S_{w,z}\setminus\{w,z\}$ for which the fastest temporal path from $z_i$ to $u$ has the biggest duration.
Denote this vertex as $z^*$. 
Then we know that all vertices in $S_{w,z}$ that are closer to $w$ than $z^*$ reach $u$ on the path that goes through $w$ and 
all the vertices in $S_{w,z}$ that are closer to $z$ than $z^*$ reach $u$ on the path that goes through $z$.
At the beginning we have also guessed the fastest path $z_1 \rightarrow w \rightarrow u$ (resp.~$z_p \rightarrow z \rightarrow u$).
Using this information we know all the fastest paths from any vertex in $S_{w,z} \setminus \{ z^* \}$ to $u$.
For all of these paths we introduce an equality constraint. 
For $z^*$ we have to determine which of the paths, through $w$ or $z$, to $u$ is the fastest.
Since we know the labels of all edges of $S_{w,z}$, with the exception of at most one edge, we can easily determine (procedure is similar as in Case~(\ref{item:FPT-uToxcase2}) from above),
which is the fastest temporal path from $z^*$ to $w$, which gives an equality constraint .
Now
we still need to add the inequality constraints for all other paths.
There are $O(k^k)$ paths between $w$ (resp.~$z$) and $u$, and a unique path from $z^*$ to $w$ (resp.~$z$).
For all paths from $w^*$ to $u$, that are not equal to the fastest, we introduce the inequality constraint.
This results in adding $O(k^k)$ inequality constraints.

Next is the case of determining the fastest path from $x$ to $y$, where $x,y \in V(G') \setminus U$.
Let $x$ be a vertex in the segment $S_{u,v} = (u,v_1,v_2, \dots, v_p, v)$ and let $y$ be a vertex in the segment $S_{w,z} = (w,z_1,z_3, \dots, z_r, z)$.
By checking the durations of the fastest paths from $x$ to every vertex in $S_{w,z} \setminus \{w,z\}$
we can determine the vertex $z_i \in S_{w,z} $, for which the duration from $x$ is the biggest.
This implies that the fastest temporal paths from $x$ to all vertices $z_1, z_2, \dots, z_{i-1}$ (resp.~$z_{i+1}, z_{i+2}, \dots, z_r$)  passes through $w$ (resp.~$z$).
Now we determine the vertex $z_j \in S_{u,v} \setminus \{u,v\}$,
for which the value of the durations of the fastest paths from it to the vertex $y$ is the biggest.
This implies that the fastest temporal paths from the vertices $v_1, v_2, \dots, v_{j-1}$ (resp.~$v_{j+1}, v_{j+2}, \dots, v_p$) to the vertex $y$ passes through $u$ (resp.~$v$).
At the beginning we have also guessed the following fastest paths  
$z_1 \rightarrow w \rightarrow u \rightarrow v_1$,
$z_1 \rightarrow w \rightarrow v \rightarrow v_p$,
$z_r \rightarrow z \rightarrow v \rightarrow v_p$ and
$z_r \rightarrow z \rightarrow v \rightarrow v_p$.
We distinguish following two cases,
first when $\{x,y\} \neq \{v_i, z_j\}$ and the second where $\{x,y\} = \{v_i, z_j\}$ (note this case encapsulates also the two cases when $x = v_i$ or $y=z_j$).
In the first case we know how the fastest temporal path from $x$ to $y$ exits the segment $S_{u,v}$ and enters the segment $S_{w,z}$.
Combining this with the guessed fastest paths, we uniquely determine the fastest temporal path from $x$ to $y$. 
For this path we add the equality constraint, and for all other paths (by similar arguing as in other cases there are $O(k^k)$ such paths) we add the inequality constraints.
In the second case when $\{x,y\} = \{v_i, z_j\}$ we do the following.
\todo{TODO}

All of the above is enough to determine the labeling $\lambda$ of $G'$. Now we have to make sure that the labeling considers also the vertices in $Z$ that we initially removed from $G$.
Remember that removed vertices form disjoint trees in $G$.
Let us denote $Z$ as the set of disjoint trees, \ie $Z = T_1 \cup T_2 \cup \cdots \cup T_t$, where $T_i$ represents one of the trees.
Since there is a unique (static) path between any two vertices $t_1, t_2$ in a tree $T_i$,
it follows that there is also a unique (therefore also the fastest) temporal path among them.
Thus determining the label of an edge in $T_i$ uniquely determines the labels on all other edges of tree $T_i$.
Let us describe now how to determine the labels on edges of an arbitrary $T_i \in Z$.
For a tree $T_i$ denote with $u_i$ the vertex in $G'$ that is a neighbor of some vertex in $T_i$, 
call it a \emph{clip vertex of the tree $T_i$}
and denote with $r_i$ the neighbor of $u_i$, \ie the root of the tree $T_i$.
By construction (by iteratively removing vertices of degree one from $G$) it follows that there can be many different trees $T_i$ that are incident to the same clip vertex $u_i \in V(G')$,
but each tree $T_i$ is incident to exactly one clip vertex $u_i \in V(G')$.
First observe that $|N_{G'}(u_i)|\geq 1$, if not also $u_i$ would be removed from $G$.
To determine the correct label of all edges of $T_i$ we use the following property.
\begin{claim}
Let $T_i$ be a tree in $Z$ and let $e_i = (u_i,r_i)$ be an edge in $G$, where $u_i \in V(G')$ is a clip vertex of $T_i$ and $r_i \in T_i$ is a root of $T_i$.
Let $v \in N_{G'}(u_i)$ be the closest vertex to $r_i$, regarding the values of $D$, \ie  $D_{r_i,v} \leq D_{r_i,w}$ for all $w \in N_{G'}(u)$.
Then the path $P^*=(r_i,u_i,v)$ has to be the fastest temporal path from $r_i$ to $v$ in $G$.
\end{claim}
\begin{proof}
Suppose that this is not true.
Then there exists a faster path $P^*_2$ from $r_i$ to $v$, that goes through the clip vertex $u_i$ of $T_i$ (as this is the only neighbor of $r_i$), through some other vertex $w \in N_{G'}(u) \setminus \{v \}$, 
and through some other path $P'$ in $G$, before it finishes in $v$, where $P'$ is at least an edge (from $w$ to $v$). 
Therefore $P^*_2= (r_i,u_i,w,P',v)$, where $d(P^*_2) \leq d(P^*)$.
Now since $D_{r_i,v} \leq D_{r_i,w}$ for all $ w\in N_{G'}(u)$ the first part of path $P^*_2$ from $r_i$ to $w$ takes at least $D_{r_i,v}$ time.
Since $v \neq w$ we need at least one more time-step (one more edge) to traverse from $w$ to reach $v$. 
So $d(P^*_2) \geq D_{r_i,v} + 1$, and so $P^*_2$ cannot be faster than $P^*$.
\end{proof}

Because $(r_i,u_i,v)$ is the fastest temporal path from $r_i$ to $v$ we can determine the label of edge $r_iu_i$ as 
$\lambda (r_i u_i) \equiv \lambda(u_i v) + 1 - D_{r_i,v} \pmod \Delta$.
Now using the algorithm for trees (see \cref{thm:deltaExact-PolyTimeTrees}), we determine labels on all edges of $T_i$.
We repeat this procedure for all trees in $Z$.
The remaining thing is to add the equality (resp.~inequality) constraints for the fastest (resp. non-fastest) temporal paths from vertices of $Z$ to all other vertices in $G'$ and vice versa.

Let us start with the case when we consider fastest paths from a vertex $t$ in some tree $T_i$ to a vertex of interest $u \in U$.
Denote with $r_i$ the root of $T_i$.
We distinguish the following two cases.
\begin{enumerate}[(i)]
    \item The clip vertex $x$ of the tree $T_i$ is not a vertex of interest.
    Let $x$ be a part of a segment $S_{w,z} = (w, z_1, z_2, \dots, z_r )$, and denote with $z_{i-1}$ and $z_{i+1}$ the neighbouring vertices of $x$, 
    where $z_{i-1}$ is closer to $w$ in $S_{w,z}$ and 
    $z_{i+1}$ is closer to $z$ in $S_{w,z}$.
    Similarly as before we know the following fastest paths $z_1 \rightarrow w \leadsto u$ and
    $z_r \rightarrow z \leadsto u$.
    Denote thew with $Q_1$ and $Q_r$ respetively.
    There are two options.
    \begin{itemize}
        \item The segment $S_{w,z}$ has no undetermined edges, with the exception of the first/last one.
        Which results in knowing all the waiting times at vertices of $S_{w,z}$ when traversing the segment.
        Then we also know that the labels of tree $T_i$ edges are determined with respect to that same edge label.
        This results in knowing the value of waiting time
        $\tau_x^{r_i,z_{i-1}}$ at vertex $x$ when traversing it from $r_i$ to $z_{i-1}$ and 
        the value of waiting time $\tau_x^{r_i,z_{i+1}}$ at vertex $x$
        when traversing it from $r_i$ to $z_{i+1}$.
        We also know the value $D_{x,u}$ and the underlying path of the fastest temporal path from $x$ to $u$ (which we determined in previous steps).
        W.l.o.g. suppose that the fastest path from $x$ to $u$ goes through $z_{i-1}$ and uses the path $Q_1$.
        Denote with $P^{-} = (r_i, x, z_{i-1}, z_1) \cup Q_1$
        and with $P^{+} = (r_i, x, z_{i+1}, z_r) \cup Q_r$.
        Then we calculate the duration $d(P^{-})$ as $d(P^{-}) = D_{x,u} + \tau_x^{r_i,z_i-1}$
        and compare it to $D_{r_i,u}$.
        If $d(P^{-}) < D_{r_i,u} $ then we stop with the calculation and determine that our input graph has no solution.
        If $d(P^{-}) = D_{r_i,u} $ then we know that $P^{-}$ is the underlying path of the fastest temporal path from $r_i$ to $u$. 
        If $d(P^{-}) > D_{r_i,u} $ then the fastest temporal path from $r_i$ to $u$ has to be $P^{+}$.
        For the fastest temporal path we introduce the equality constraint, 
        for all other paths we introduce the inequality constraints.
        By similar arguing as in cases above, we introduce $O(k^k)$ inequality constraints.
        \item The segment $S_{w,z}$ has an extra undetermined edge $p$. If $p \cap \{x\} = \emptyset$, 
        we can proceed with the same approach as above.
        So suppose now that $p = x z_{i+1}$. 
        Then, from knowing that $p$ is and undetermined edge we conclude that
        all fastest temporal paths from $x$ to any vertex of interest $u'$ go through the edge $z_{i-1}x$ (not trough $p$).
        Now, if the edges of tree are determined with respect to the label of the edge $z_{i-1}x$,
        we use the same approach as above to determine the fastest temporal path from $r_i$ to $u$.
        Therefore, suppose that the edges of the tree $T_i$ are determined with respect to the label of the edge $p$.
        Which means that $D_{r_i, z_{i+1}} < D_{r_i, z_{i-1}}$.
        We want to now determine if the fastest temporal path from $r_i$ to $u$ is of form 
        $r_i \rightarrow x \rightarrow z_{i-1} \rightarrow \dots \rightarrow w \leadsto u$ or
        $r_i \rightarrow x \rightarrow z_{i+1} \rightarrow \dots \rightarrow z \leadsto u$.
        We do the following.
        Denote with $c$ the value $c = D_{r_i z_{i-1}} + D_{x u} + 1$.
        We now claim the following, note that we do not know what is the fastest temporal path from $r_i$ to $x_{i-1}$. 
        It can be of the form $P^- = (r_i,x,z_{i-1})$,
        or of the form $P^+ = (r_i,x, z_{i+1}, z_{i+2}, \dots, z) \cup Q \cup (w, z_1, \dots, z_{i-1}$, where $Q$ is some path from $z$ to $w$.
        Denote with $R^x$ the underlying path of the fastest temporal path from $x$ to $u$,
        and with $R^{i-1}$ the underlying path of a fastest temporal path from $z_{i-1}$ to $u$.
        Note $R^{i-1} \subset R^x$.
        Similarly,
        denote with $R^{i+1}$ the underlying path of the fastest temporal path from $z_{i+1}$ to $u$, 
        for which we know it goes through the vertex $z$.
        \begin{itemize}
            \item  If $c < D_{r_i,u}$, then we have a contradiction and we stop with the calculation.\\
            This is true since we have found a temporal path from $r_i$ to $u$, with faster duration than the fastest temporal path from $r_i$ to $u$,
            which cannot happen.
            \item  If $c = D_{r_i,u}$, then 
            the fastest temporal path is of form $r_i \rightarrow x \rightarrow z_{i-1} \rightarrow \dots \rightarrow w \leadsto u$.\\
            We have two options, first when the fastest temporal path from $r_i$ to $z_{i-1}$ is $P^-$.
            In this case we have determined that $P^- \cup R^{i-1}$ is the fastest temporal path from $r_i$ to $u$.
            In the second case we suppose that the fastest temporal path from $r_i$ to $z_{i-1}$ is $P^+$.
            But then the duration of the path $P^+ \cup R^{i-1}$ from $r_i$ to $u$ equals the duration of the fastest path from $r_i$ to $u$. But note that $P^+ \cup R^{i-1}$ is actually not a path but a walk, since there is repetition of edges between $w$ and $z_{i-1}$,
            therefore it includes a path from $r_i$ to $u$, which is even faster,
            a contradiction.
            Therefore we get that in this case $P^-$ is always the underlying path of the fastest path from $r_i$ to $z_{i-1}$.
            And the fastest path from $r_i$ to $z_{i-1}$ is $P^- \cup R^{i-1}$.
            \item If $c > D_{r_i,u}$, then 
            the fastest temporal path is of form $r_i \rightarrow x \rightarrow z_{i+1} \rightarrow \dots \rightarrow z \leadsto u$.\\
            We again have two options.
            First when the fastest temporal path from $r_i$ to $z_{i-1}$ is $P^-$.
            In this case we easily deduce that $P^- \cup R^{i-1}$ is not the underlying path of the fastest temporal path from $r_i$ to $u$.
            And therefore it follows that the the underlying path of the fastest temporal path from $r_i$ to $u$ is  $(r_i, x, z_{i+1}) \cup R^{i+1}$.
            In the second case, suppose that $P^+$ is the underlying path of the fastest temporal path from $x_i$ to $z_{i-1}$.
            We want to now prove that the fastest temporal path from $r_i$ to $u$ travels through vertices $z_{i+1}, z_{i+2}, \dots z$.
            Suppose for the contradiction that this is not true. 
            Then $S = (r_i, x, z_{i-1}) \cup R^{i-1}$ is the underlying path of the fastest temporal path from $r_i$ to $u$.
            Then we get that the duration $d(S)$ of $S$ equals to $D_{r_i,u}$.
            Let $D(r_i,z_{i-1},S)$ be the duration of the temporal path from $r_i$ to $z_{i-1}$ along the path $S$.
            By the definition we get that $d(S) = D(r_i,z_{i-1},S) + D_{x,u} - 1$.
            From this it follows that $D(r_i,z_{i-1},S) = D_{r_i, z_{i-1}}$, which is in contradiction with our assumption.
            Therefore we get that in this case 
            $(r_i, x, z_{i+1}) \cup R^{i+1}$ is always the underlying path of the fastest path from $r_i$ to $z_{i-1}$.
        \end{itemize}
       In all of the cases, we have uniquely determined the underlying path of the fastest temporal path from $r_i$ to $u$,
       which gives us an equality constraint.
       For all other paths we add the inequality constraints. 
       There are $O(k^k)$ of paths like this.
    \end{itemize}
    We also know the waiting times at the clip vertex $x$ coming from $z_{i-1}$ and $z_{i+1}$. Therefore we can determine the fastest temporal path $P$ from $u$ to $r_i$.
    Again this path is the one that is then extended to the fastest path from $u$ to all other vertices in $T_i$,
    which gives rise to $|T_i|$ equality constraints.
    We iterate through all other temporal paths and produce the inequality constraints.
    There are $O(k^k)$ of them for each vertex in $T_i$.
    
    \item The clip vertex $w$ of the tree $T_i$ is a vertex of interest.
    Note that since we have determined the label of the edge $w r_i$, we also know the waiting times 
    at vertex $u$ coming from each of its neighbor $v_i$ and finishing in $r_i$. 
    At the beginning we have also determined the fastest paths from $u$ to all $v_i \in N_{G'}(w)$.
    Therefore we know the path $P$ that is the fastest from $u$ to $r_i$. 
    Once we have determined $P$ we also know that the fastest path from $u$ to any other vertex in $T_i$ uses $P$ at the beginning, to go from $u$ to $r_i$.
    Therefore we have determined the fastest paths from $u$ to all other vertices in $T_i$,
    which gives rise to $|T_i|$ equality constraints.
    We iterate now through all other $u \leadsto w \rightarrow r_i$ paths and introduce inequality constraints for them.
    There are $O(k^k)$ such paths.
\end{enumerate}

In the case when we want to determine the fastest path from a vertex $t$ in some tree $T_i$ to a vertex of interest $u \in U$,
we use similar arguing as above.
The procedure produces one equality constraint and $O(k^k)$ inequality ones.

The only two cases to consider are the cases when we want to determine the fastest paths from some $x \in V(G)$, that is not a vertex of interest, to a vertex $t$ in some tree $T_i$, and vice versa.
We again need to consider two cases, one where 


\bibliography{bibliography}	


\clearpage

\appendix

\section{FPT, Old proof -- need to move useful stuff above}
Let us first observe some properties.

Suppose that $Q = (u,v_2,v_3, \dots, v_p, v)$, where $p \geq 2$ is an undetermined segment in $G'$, between vertices of interest $u,v \in U$,
let $P$ be the underlying path of a fastest temporal path from $u',v' \in U$.
Let $P$ be such path that there exists a vertex $x \in P \setminus \{u,v\}$ of degree two (\ie $x\notin U$).

Since $x$ is an inner vertex of $P$ and is of degree two, then we know the labels of its two edges incident to it.
Denote with $\ell_x$ the edge incident to $x$ that is closer to $u'$ on the path $P$, and with $r_x$ the edge that is closer to $v'$ on $P$.
Let $y$ be an inner vertex on the path $Q$.
Similarly, denote with $\ell_y$ the edge incident to $y$ that is closer to $u$ on the path $Q$, and with $r_y$ the edge that is closer to $v$ on $Q$.
Observe the following. 
To reach $y$ from $x$ there are $4$ different options, at the start path uses either edge $\ell_y$ or $r_y$, 
and at the end path must use either $\ell_x$ or $y_x$.
Since this is true for any path (hence any temporal path), it holds also for the fastest temporal path from $x$ to $y$.

\begin{claim}\label{claim:FPT-unlabeldPaths-012}
    Suppose that the fastest temporal path from $x$ to $y$ uses at the beginning edge $\ell_x$.
    We now claim that if the fastest temporal path from $x$ to $y$ uses edge $\ell_y$ (resp.~$r_y$) as the last edge,
    then the fastest temporal path from $x$ to any other vertex of $Q$, that is closer to $u$ (resp.~$v$) 
    uses the same edge $\ell_y$ (resp.~$r_y$) as the last edge and $\ell_x$ as the first edge.
    Similarly holds if the first edge of the fastest temporal path is $r_x$.
\end{claim}
\begin{proof}
    Suppose that $y, y'$ are two vertices of $Q$, where the distance from $u$ to $y'$, in $Q$, is shorter than the distance from $u$ to $y$.
    Suppose also that the fastest temporal path from $x$ to $y$ uses the edges $\ell_x$ and $\ell_y$, and passes through $y'$. Denote this path as $P_{x,y}$.
    Denote with $P_{x,y}^{y'}$ the subpath of the temporal path $P_{x,y}$ from $x$ to $y'$
    Now observe the following possibilities. %
    \\
    First, suppose that a fastest temporal path from $x$ to $y'$ uses the edges $\ell_x$ and $r_y$, denote it with $P_{x,y'}^1$.
    Then the duration of $d(P_{x,y}^{y'}) \geq  d(P_{x,y'}^1)$.
    But since $P_{x,y'}^1$ passes through vertex $y$, the duration from $x$ to $y$ on $P_{x,y'}^1$ is strictly smaller than $d(P_{x,y'}^1)$, and conversely 
    $d(P_{x,y})$ is strictly greater than $d(P_{x,y}^{y'})$,
    therefore at vertex $y$ we get that 
    $d(P_{x,y})$ is strictly bigger than the duration of the subpath of $P_{x,y'}^1$ from $x$ to $y$, which is not possible. %
    \\
    Second, suppose that a fastest temporal path from $x$ to $y'$ uses the edges $r_x$ and $\ell_y$, denote it with $P_{x,y'}^2$.
    Therefore $d(P_{x,y'}^2) \leq d(P_{x,y}^{y'})$.
    To get the duration of the initial $P_{x,y}$ we can split the problem into calculating the duration from $x$ to $y'$ and the duration from $y'$ to $y$, plus the waiting time at vertex $y'$.
    Since there is a unique path from $u$ to $y'$ and then to $y$, the waiting time at $y'$ and the duration of the path from $y'$ to $y$ is independent of the choice of path from $x$ to $y'$. 
    Therefore, from the fact that $P_{x,y}$ is a fastest temporal path from $x$ to $y$ we get that $d(P_{x,y'}^2) \geq d(P_{x,y}^{y'})$.
    All together we conclude that $d(P_{x,y}^{y'})$ is a fastest temporal path from $x$ to $y'$.%
    \\
    Third, suppose that a fastest temporal path from $x$ to $y'$ uses the edges $r_x$ and $r_y$, denote it with $P_{x,y'}^3$.
    Then the subpath of $P_{x,y'}^3$ from $x$ to $y$ would be strictly faster than $P_{x,y}$, which is not possible.
\end{proof}
%
Using the results from above we can now prove the following.
\begin{lemma}\label{lem:FPT-DetermineLabelsOnUnlabeledNonFastestPaths}
    Suppose that $Q = (u,v_2,v_3, \dots, v_p, v)$, where $p \geq 2$ is an undetermined segment in $G'$, between vertices of interest $u,v \in U$,
    let $P$ be the underlying path of a fastest temporal path from $u',v' \in U$.
    Let $P$ be such path that there exists a vertex $x \in P \setminus \{u,v\}$ of degree two (\ie $x \notin U$).
    Then we can precisely determine the labels on every edge of $Q$, but one, with respect to the label $c_{u'v'}$ of the first edge of path $P$.
\end{lemma}
\begin{proof}
    %To prove this we use \cref{claim:FPT-unlabeldPaths-01} and \cref{obs:FirstLabelAndDuration}.
    Let us denote with $y$ the vertex of $Q$, for which the duration from $x$ to $y$ is the biggest, among vertices of $Q$,
    more precisely $y$ is a vertex corresponding to $max_{v_i \in Q} D_{x,v_i}$.
    Denote 
    with $\ell_x$ the edge incident to $x$ that is closer to $u'$ on path $P$, and with $r_x$ the edge that is closer to $v'$ on $P$.
    Similarly, denote with $\ell_y$ the edge incident to $y$ that is closer to $u$ on the path $Q$, and with $r_y$ the edge that is closer to $v$ on $Q$.
    Suppose that the fastest temporal path $P$ from $x$ to $y$ uses first edge $\ell_x$ and then edge $\ell_y$.
    Then it follows from \cref{claim:FPT-unlabeldPaths-01} that all fastest temporal paths from $x$ to vertices between $u$ and $y$ use as a first edge $\ell_x$ and go through vertex $u$. 
    Now using \cref{obs:FirstLabelAndDuration} it follows that we can determine labels on all edges between $u$ and $y$.

    Denote now with $y'$ the neighbor of $y$ in $Q$, that is closer to $v$ than $u$. 
    We know that we reach $y'$ starting from $x$ the fastest, when the path passes through vertex $v$.
    There are two options how the fastest temporal path from $x$ to $y'$ departs from $x$. Either it first uses edge $\ell_x$ or edge $r_x$. 
    Once we know which of these two options happens, we can determine all the labels on edges between $v$ and $y'$.

    In total we have uniquely determined labels of all edges of $Q$, with respect to the labeling of edges in $P$ (\ie with respect to the label of the first edge on $P$), with the exception of edge $yy'$.
\end{proof}

Up to now we have determined the labels (the constraints for labels) of edges of $G'$. We will now extend the labeling to the initial graph $G$. 
Recall, $G'$ was created from $G$ by iteratively removing all vertices of degree $1$ from $G$.

Note that vertices from $V(G) \setminus V(G')$ form disjoint paths in $G$. 
Each such path is incident to exactly one vertex from $G'$.
 
Let $P' = (v_1, v_2, \dots, v_p=s)$ be a path in $G$ on vertices from $V(G) \setminus V(G')$  (\ie path on vertices that were removed from $G$)
and let $u$ be a neighbor of $s$ in $G$,
therefore $P=(v_1, v_2, \dots, v_p=s, u)$ is an induced path in $G$.
First observe that $|N_{G'}(u)|\geq 1$, if not also $u$ would be removed from $G$ and $G$ would be a tree, for which we can solve the problem in polynomial time (see~\cref{thm:deltaExact-PolyTimeTrees}).

To determine the correct label of all edges of $P$ we use the following property.
\begin{claim}
Let $P = (v_1, v_2, \dots, v_p=s)$ be a path in $G$ on vertices from $V(G) \setminus V(G')$
and let $u$ be a neighbor of $s$ in $G$.
Let $v_i \in N_{G'}(u)$ be the closest vertex to $v$, regarding the values of $D$, \ie  $D_{s,v_i} \leq D_{s,v_j}$ for all $v_j \in N_{G'}(u)$.
Then the path $P^*=(s,u,v_i)$ has to be the fastest temporal path from $s$ to $v_i$ in $G$.
\end{claim}
\begin{proof}
Suppose that this is not true.
Then there exists a faster path $P^*_2$ from $s$ to $v_i$, that goes through $u$ (as this is the only neighbor of $s$), through some other vertex $v_j \in N_{G'}(u) \setminus \{v_i \}$, 
and through some other path $P'$ in $G$, before it finishes in $v_i$, where $P'$ is at least an edge (from $v_j$ to $v_i$). 
Therefore $P^*_2= (s,u,v_j,P',v_i)$, where $d(P^*_2) \leq d(P^*)$.
Now since $D_{s,v_i} \leq D_{s,v_j}$ for all $v_j \in N_{G'}(u)$ the first part of path $P^*_2$ from $s$ to $v_j$ takes at least $D_{s,v_i}$ time.
Since $v_i \neq v_j$ we need at least one more time-step (one more edge) to traverse from $v_i$ to reach $v_j$. 
So $d(P^*) \geq D_{s,v_i} + 1$, and so $P^*_2$ cannot be faster than $P^*$.
\end{proof}

Because $(s,u,v_i)$ is the fastest temporal path from $s$ to $v_i$ we calculate the label of edge $su$ as $\lambda (s,v_i) \equiv \lambda(u v_i) + 1 - D_{s,v_i} \pmod \Delta$.
Since $P$ is a path, the fastest temporal path from all vertices of $P$ to $u$ is the temporal path along $P$, therefore using \cref{obs:FirstLabelAndDuration} and the value of the edge $su$, we can determine labels of all edges in $P$.

We repeat the above procedure for paths in $G$ that we removed in Step 1, and we end up with the labeling $\lambda$ of the whole graph.

\begin{comment}
Now we have to make sure that all the new labels satisfy $D$, to do this we calculate the matrix $D'$ of fastest temporal paths among all pairs of vertices in $(G, \lambda)$ and compare the results with $D$, if any values differ, we stop and return a \textsc{No} answer, if $D = D'$ we return the labeling $\lambda$.

To calculate $\lambda (s,v_i)$ we need $O(|N_{G'}(u)|) = O(n)$ time, therefore to calculate all labels of edges incident to vertices of degree one in $G$, we need $O(n^2)$ time, calculating $D'$ takes $O(n^3\log n)$ time (see \cref{lemma:calculatingD}), therefore the whole step is performed in polynomial time.
\end{comment}
%
%
\subsubsection*{Step 3(a) of the FPT algorithm}
Using all of the above results we proceed in the following way.
For a partial ILP instance $I$ we 
select an inner vertex $x$ of every fastest temporal path among vertices of interest $u'$ and $v'$,
and for every endetermined path among vertices of interest $u$ and $v$ we find such $y$ and $y_1, y_2$ from this path, that $y$ is furthest away from $x$ and $y_1, y_2$ are its neighbors.
We then enumerate $4$ possible ways that a fastest temporal path starts from $x$ and reaches $y$, 
and then $2$ additional ways, how a fastest temporal path starts from $x$ to $y_1$ (resp.~$y_2$).
For each of the possibilities we then precisely determine labels of the undetermined path, up to the constant of the fastest $u',v'$ temporal path.

Since there are at most $|U|^2$ fastest temporal paths among vertices of interest, we create at most $4|U|^2$ extra instances for each undetermined path.
Therefore one partial ILP instance $I$ is further extended into $O(|U|^3)=f(k)$ \todo{$O(2^k)$} final ILP instances $I_I$.

\subsubsection*{Step 3(b) of the FPT algorithm}
Using results from \cite{Lenstra1983Integer} we know that an ILP instance can be solved in FPT time with respect to the number of variables. In the case when the input instance has a solution, algorithm also provides us with one.

One final ILP instance $I_I$, that was built by our algorithm, has $f(k)$ variables. 
Therefore one $I_I$ can be solved in FPT time with respect to $k$.

We have build $f(k)$ different final ILP instances. If none of the instances is a \textsc{Yes} instance, then we conclude that there is no labeling $\lambda$ of $G$, that would satisfy the distances of the matrix $D$,
and therefore $D$ is a \textsc{No} instance of \deltaExact.
If there is a final ILP instance $I_I$, that is a \textsc{Yes} instance, then we take a solution $S$, provided by the algorithm from \cite{Lenstra1983Integer}, and determine the corresponding labeling $\lambda$ of $G$.
Note, the minimum number of solutions of $I_I$ is at least $\delta$, since the graph is periodic and adding the same constant to every edge in the graph, dose not change the duration of any temporal path in $(G, \lambda)$.

\section{
\texorpdfstring{\deltaExact\ is (NOT) polynomial-time solvable}
{\textsc{Simple Delta-TGR} is polynomial-time solvable}
}\

$G$ can be deduced easily from $D$.

	Let $G$ be the underlying graph of matrix $D$ and let $v$ be an arbitrary vertex in $G$, we want to determine the labeling $\lambda: E(G) \rightarrow [\Delta]$, that assigns exactly one label to each edge in $G$ and satisfies the distances from $D$.
	To do so we observe some properties of the problem.
	
\subsection*{Inequalities}
	
	Let $v \in V$ be an arbitrary vertex in $V(G)$ and let $u \in V \setminus \{v\}$ be a vertex, that is at distance at least $2$ from $v$, in the underlying graph $G$.
	Let $\mathcal{P}_{u}^v$ be a set of all $P_3$ paths on $3$ vertices in $G$, of form $(v_1, v_2, u)$, where $v_1, v_2 \in V(G)$, \st $v_1 v_2, v_2 u \in E(G)$ and $d(v,v_1) < d(v,v_2) < d(v,u)$. 
	Intuitively, we can imagine the set $\mathcal{P}_{u}^v$ as the set of all incoming temporal $P_3$ paths to vertex $u$, with respect to starting from vertex $v$.
	
	Let $P = (v_1,v_2,u)$ be an arbitrary path from $\mathcal{P}_u^v$, denote with $\lambda_1, \lambda_2$ the labels of edges $(v_1,v_2)$ and $(v_2, u)$, respectively.
	By the definition of $\mathcal{P}_u^v$ it follows that $v_1, v_2$ are reached by $v$ faster than $u$ is reached by $v$, therefore we get the following inequality:
	\begin{equation}\label{eq:ineq-uv-onePath}
	    \lambda_2 \geq_\Delta \lambda_1 + d(v,u)-d(v,v_2) \pmod \Delta
	\end{equation}
 \todo[inline]{GM: WHY??? THIS IS WRONG}
	We repeat the above procedure for every path $P \in \mathcal{P}$ and collect all the inequalities to a set $S_u^v$.
	Similarly we proceed for every pair of vertices $u$ and $v$.
	At the end we get the collection of sets, each containing inequalities of the form \cref{eq:ineq-uv-onePath}.
	
	\begin{lemma}
	\label{claim:exact-setsOfInequalities}
    Let $D$ be \textsc{YES}-instance of \textsc{Simple $\Delta$-TGR}, and let 
$(G,\lambda,\Delta)$ be a certificate for $D$, where $\lambda:E\rightarrow \{1,2,\ldots,\Delta\}$ a simple periodic labeling of $G$. Then, for every pair $u,v$ of non-adjacent vertices:
	\begin{enumerate}
	    \item \label{itemize:Claim-exact-AtLeastOneEquality-1} $\lambda$ satisfies all the inequalities of the set $S_u^v$, and 
	    \item \label{itemize:Claim-exact-AtLeastOneEquality} at least one of the inequalities of $S_u^v$ is an equality.
	\end{enumerate}
	\end{lemma}

    \begin{proof}
    By the definition of $\mathcal{P}_u^v$, every fastest temporal path from $v$ to $u$ must reach $u$ using one of the $P_3$ paths of $\mathcal{P}_u^v$.
    Let $P$ be an arbitrary temporal path from $v$ to $u$. 
    An arbitrary path $P \in \mathcal{P}_u^v$ is either a part of the fastest path from $v$ to $u$, which means that the path arrives to $u$ at the time equal to $d(v,u)$,
    or it is not a part of the fastest path, which means it arrives to $u$ at some time greater than $d(v,u)$.
    This is how the inequalities from $S_u^v$ are constructed. 
    \todo{GM: WHY? WRONG} 
    Since the fastest path from $v$ to $u$ has to be of duration $d(v,u)$, at least one of the inequalities from $S_u^v$ has to be an equality.    
    \end{proof}
    
    
    \paragraph{Manipulation of inequalities modulo $\Delta$.}
    Suppose we have a path $P$ on three vertices, of form $(v_1, v_2, u)$,
    and suppose $d(v,v_1) < d(v,v_2) < d(v,u)$ for some $v \in V \setminus P$, with $d(v,v_2) = a, d(v,u)=b$. 
    Let us denote with $\lambda_1 = \lambda(v_1 v_2), \lambda_2 = \lambda(v_2 u)$ and $k = b-a$, where $1 \leq k \leq \Delta$ as all edge labels have values from $1$ to $\Delta$.
    Then by \cref{eq:ineq-uv-onePath} we know that 
    \begin{equation}\label{eq:inequalities-properties-example}
        \lambda_2 \geq_\Delta \lambda_1 + k \pmod \Delta.
    \end{equation}
    Note that if $\lambda_2 = \lambda_1$, the duration of temporal path $(P, \lambda) = ((v_1v_2, \lambda_1), (v_2u, \lambda_2))$ is exactly $\Delta+1$,
    therefore $\lambda_1 = \lambda_2$ satisfies \cref{eq:inequalities-properties-example}.
    We can rewrite \cref{eq:inequalities-properties-example} as
    \begin{equation*}
        \lambda_2 \in 
        \{
        \lambda_1 + k, \lambda_1 + (k + 1), \dots , \lambda_1 + (\Delta - 1), \lambda_1 + \Delta 
        \} \pmod \Delta.
    \end{equation*}
    Suppose now we have the following two inequalities
    \begin{align*}
        & \lambda_2 \geq_\Delta \lambda_1 + k \pmod \Delta, \\
        & \lambda_2 \geq_\Delta \lambda_1 + k' \pmod \Delta,
    \end{align*}
    for some $k' > k$, then the set of all possible solutions satisfying both equations is the following
        \begin{equation*}
        \lambda_2 \in 
        \{
        \lambda_1 + k', \lambda_1 + (k' + 1), \dots , \lambda_1 + (\Delta - 1), \lambda_1 + \Delta 
        \} \pmod \Delta.
    \end{equation*}
    
    Observe now that without the condition~\ref{itemize:Claim-exact-AtLeastOneEquality} of \cref{claim:exact-setsOfInequalities}, all inequalities of condition~\ref{itemize:Claim-exact-AtLeastOneEquality-1} can be satisfied by setting the label of every edge of the graph to be the same.
    
    \subsection*{Cycles}
    \begin{definition}[Travel delays]
    Let $e_1=uv$ and $e_2=vz$ be two incident edges with $e_1 \cap e_2 = v$.
    We define the \emph{travel delay} from $u$ to $z$ at vertex $v$, denoted with $\tau_v^{uz}$,
    as the difference of the labels of $e_2$ and $e_1$, where we subtract the value of the label of $e_1$ from the label of $e_2$,  modulo~$\Delta$.
    More specifically:
    \begin{equation}\label{eq:def-VertexWaitingTimes}
       \tau_v^{uz} \equiv \lambda (e_2) - \lambda(e_1) \pmod \Delta.
    \end{equation}
    Similarly, $\tau_v^{zu} \equiv \lambda (e_1) - \lambda(e_2) \pmod \Delta$.
    \end{definition}
    Intuitively, the value of $\tau_v^{uz}$ represents how long a temporal path waits at vertex $v$ when first taking edge $e_1=uv$ and then edge $e_2 = vz$.
    
    \begin{observation}
    For any two incident edges $e_1 = uv$ and $e_2 = vz$ on vertices $u,v,z \in V$, with $e_1 \cap e_2 = v$, we have $\tau_v^{zu} = \Delta - \tau_v^{uz} \pmod \Delta$.
    \end{observation}
    
    \begin{proof}
        Let $e_1 = uv$ and $e_2 = vz$ be two edges in $G$ for which $e_1 \cap e_2 = v$. 
        By the definition $\tau_v^{uz} \equiv \lambda (e_2) - \lambda(e_1) \pmod \Delta$ and $\tau_v^{zu} \equiv \lambda (e_1) - \lambda(e_2) \pmod \Delta$.
        Summing now both equations we get $\tau_v^{uz} + \tau_v^{zu} \equiv \lambda(e_2) - \lambda(e_1) + \lambda (e_1) - \lambda(e_2) \pmod \Delta$, and therefore $\tau_v^{uz} + \tau_v^{zu} \equiv 0 \pmod \Delta$, which is equivalent as saying $\tau_v^{uz} \equiv - \tau_v^{zu} \pmod \Delta$ or $\tau_v^{zu} = \Delta - \tau_v^{uz} \pmod \Delta$.
    \end{proof}

    \begin{observation} \label{obs:exact-VertexSum}
    Let $v$ be a vertex of $G$, and let $N(v)=\{u_1, u_2, \ldots, u_k\}$. Then 
    \begin{equation*}
        \sum_{i = 1}^k \tau_{v}^{u_{i}v_{i+1}} \equiv  0 \pmod \Delta,
    \end{equation*}
    where the indices of vertices $u_i$ are taken modulo $k$.
    \end{observation}

    \begin{proof}
        tbd
    \end{proof}


\begin{observation} \label{obs:exact-CyclesSum}
    Let $C = (v_1, v_2, \dots, v_k)$ be a cycle in $G$.
    Then 
    \begin{equation*}
        \sum_{i = 1}^k \tau_{v_i}^{v_{i-1}v_{i+1}} \equiv  0 \pmod \Delta,
    \end{equation*}
    where the indices of vertices $v_i$ are taken modulo $k$ (\ie $v_{-1}=v_{k-1}, v_0 = v_k, v_1 = v_{k+1}, \dots$).
    \end{observation}
    
    
    \begin{proof}
        Using \cref{eq:def-VertexWaitingTimes} and the fact that the indices of vertices $v_i$ are taken modulo $k$, we get the following:
        \begin{align*}
            &\sum_{i = 1}^k \tau_{v_i}^{v_{i-1}v_{i+1}}  \equiv  
            \sum_{i = 1}^k  ( \lambda(v_i v_{i+1}) - \lambda (v_{i-1} v_i) ) \pmod \Delta \equiv \\
            &\lambda(v_1 v_2) - \lambda(v_{k}v_1) + \lambda(v_2 v_3) - \lambda(v_2 v_1) + \dots + \lambda(v_k v_1) - \lambda(v_{k-1}v_k) \pmod \Delta.
        \end{align*}
    \end{proof}
    
    \begin{claim}
    labeling $\lambda$ is a solution for the \textsc{Exact Realization $(\Delta)$} if and only if
    for every cycle $C$ of $(G,\lambda)$ it holds that the sum of the travel delays along $C$ is $0$ modulo $\Delta$.
    \end{claim}
    
    \begin{proof}
        TBD
    \end{proof}
    
\begin{theorem}
	\textsc{Exact Realization $(\Delta)$} can be solved in polynomial time on paths, cycles, trees and stars.
\end{theorem}

\begin{proof}[Idea:]
    For a labeling $\lambda$ we know that it has to satisfy \cref{claim:exact-setsOfInequalities}.
    There are at most $n^2$ different sets $S_u^v$, each containing at most $n^2$ inequalities. 
    Therefore we have $O(n^4)$ inequalities to satisfy with at least of $n^2$ of them being equalities (one for each set $S_u^v$. 
    
    Besides that we can also use the fact from \cref{obs:exact-CyclesSum}, which produces some extra restrictions on all inequalities.
    We believe we can create some ``base of cycles'' over $G$ that would help us use \cref{obs:exact-CyclesSum} efficiently.
\end{proof}


\end{document}