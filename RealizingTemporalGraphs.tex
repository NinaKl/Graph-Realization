\documentclass[11pt,a4paper]{article}

\usepackage[english]{babel}

\usepackage{fullpage}

\usepackage{amsmath}
\let\proof\relax
\let\endproof\relax


\usepackage{amsthm}
\usepackage{amssymb}
\usepackage{amsfonts}
\usepackage{mathrsfs}
\usepackage{wrapfig} %wrap the text around the picture

\usepackage{thmtools} 
\usepackage{thm-restate}

\usepackage{caption} %for subfigure - join multiple figures and add captions
\usepackage{subcaption}
\usepackage{changepage} %inside figure we can put \begin{adjustwidth}{-1cm}{-1cm} \end{adjustwidth} and we can ignore the margins
\usepackage{mathtools} % write text [under]{over} arrow $\xrightarrow[\text{world}]{\text{hello}}$

\usepackage{graphicx}
\graphicspath{{Figures/}}

%\usepackage{enumitem} %enumerate 
\usepackage{enumerate}
\usepackage{todonotes}
\newcommand{\todonkl}[2][]{\todo[color=red!100!green!33,#1]{NKL:\\ #2}} %todo note that starts with NKL:
\setlength{\marginparwidth}{3.7cm} %setting the length of todonote

\newtheorem{theorem}{Theorem}
\newtheorem{observation}{Observation}
\newtheorem{lemma}[theorem]{Lemma}
\newtheorem{corollary}[theorem]{Corollary}
\newtheorem{claim}[theorem]{Claim}
\newtheorem{conjecture}[theorem]{Conjecture}
\theoremstyle{remark}
\newtheorem{remark}[theorem]{Remark}
\theoremstyle{definition}
\newtheorem{definition}[theorem]{Definition}
\newtheorem{example}[definition]{Example}
\usepackage{comment}

\usepackage{algorithm}
\usepackage[noend]{algpseudocode}

%font encoding
\usepackage[T1]{fontenc}
\usepackage[utf8]{inputenc}
\usepackage{lmodern}

\usepackage{hyperref}

\usepackage[capitalise,nameinlink, noabbrev]{cleveref}
\crefname{claim}{Claim}{Claims}

\usepackage{authblk}

%\usepackage{natbib}

\newcommand{\ie}{i.\,e.,\ }
\newcommand{\st}{s.\,t.,\ }
\newcommand{\NP}{\textrm{NP}}
\newcommand{\APX}{\textrm{APX}}
\newcommand{\FPT}{\textrm{FPT}}
\newcommand{\XP}{\textrm{XP}}

\newcommand{\NN}{\mathbb{N}}
\newcommand{\ZZ}{\mathbb{Z}}
\newcommand{\RR}{\mathbb{R}}
\newcommand{\QQ}{\mathbb{Q}}


%%%%%%%%%%%%%%%%%%%%%%%%%%%%%%%% Defining a problem
%%%%%%%%%%%%%%%%%%\problemdef{NAME} {Input} {Output}
\usepackage{tabularx}
\newcommand{\problemdef}[3]{
	\begin{center}
		\begin{minipage}{0.95\textwidth}
			\noindent
			#1
			\vspace{5pt}\\
			\setlength{\tabcolsep}{3pt}
			\begin{tabularx}{\textwidth}{@{}lX@{}}
				\textbf{Input:}& #2 \\
				\textbf{Question:}& #3
			\end{tabularx}
		\end{minipage}
	\end{center}
}

\title{Realizing Temporal Graphs}
\author{Nina Klobas, George B. Mertzios, Hendrik Molter, Paul G. Spirakis}


\bibliographystyle{abbrv}
\begin{document}
	\maketitle

\section{Preliminaries}

\begin{definition}
	Let $P = ((e_1, \lambda_1), (e_2,\lambda_2), \dots , (e_k, \lambda_k))$ be a temporal path from $u$ to $v$, then the duration of $P$, denoted as $d(P)$ is $\lambda(e_k) - \lambda(e_1) + 1$.
\end{definition}

We denote with $d(u,v)$ the duration of the \emph{fastest} temporal path from $u$ to $v$.
	
	
\problemdef{\textsc{Exact Realization $(\Delta, k)$}}
{A matrix $D$ of exact durations of fastest temporal paths among each pair of vertices, and a period $\Delta$.}
{Does there exist a labeling $\lambda$ of the underlying graph $G$ of $D$, such that every edge receives exactly one label and $d(v_i,v_j) = D_{i,j}$, for any pair of vertices $v_i, v_j \in V(G)$.}

\problemdef{\textsc{Upper-bounded Realization $(\Delta, k)$}}
{A matrix $D$ of exact durations of fastest temporal paths among each pair of vertices, and a period $\Delta$.}
{Does there exist a labeling $\lambda$ of the underlying graph $G$ of $D$, such that every edge receives exactly one label and $d(v_i,v_j) \leq D_{i,j}$, for any pair of vertices $v_i, v_j \in V(G)$.}


\section{Results}

\begin{theorem}
	\textsc{Exact Realization $(\Delta, k)$} can be solved in polynomial time on paths, cycles, trees and stars.
\end{theorem}

\begin{proof}[Idea:]
	Starting from a fixed vertex we can fix the temporal path to all/some other vertices in the graph. This allows us to determine the value of labels on edges, with respect to each other.
	We then fix one label as $1$ and all other labels follow.
	Produced labeling is unique, up to $\Delta$ (\ie we get $\Delta$ ``equivalent'' labelings).
	
	Exact proof TBD.
\end{proof}

\begin{theorem}
	\textsc{Exact Realization $(\Delta, k)$} is \NP-hard on general graphs.
\end{theorem}

\begin{proof}
	Reduction from 3-SAT.
	
	Idea for gadgets, see ``Whiteboards/2022-11-03-Hendrik-Nina-Paul.pdf'', details TBD.
\end{proof}

%\bibliography{bibliography}	
\end{document}